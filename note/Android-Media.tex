\documentclass[a4paper,10pt,twoside]{article}% a4paper - A4纸 11pt -字体 twoside -双面 openany -新章节可在偶数页开始
%\usepackage{CJKutf8}%Chinese layout
\usepackage{xeCJK}
\setCJKmainfont{AR PL UKai CN}
\setCJKmonofont{AR PL UKai CN}
%\setCJKfamilyfont{AR PL UKai CN}{kai}
\setmainfont{DejaVu Sans}
\usepackage{underscore}
\pagestyle{empty} %不显示页码
%\input{pagegeometry}%自定义的文件,设置页面参数
%\usepackage{amsmath,amsthm,amsfonts,amssymb,bm} % 数学宏包
%\usepackage{graphicx,psfrag}                    % 图形宏包
\usepackage{makeidx}                            % 建立索引宏包
\usepackage{verbatim}
\usepackage{comment}
\usepackage{listings}%源代码宏包,such as C/C++ code
\usepackage{xcolor}
\lstset{language=C++}%这条命令可以让LaTeX排版时将C++键字突出显示
\lstset{breaklines}%这条命令可以让LaTeX自动将长的代码行换行排版
\lstset{extendedchars=false}%这一条命令可以解决代码跨页时,章节标题,页眉等汉字不显示的问题
\lstset{numbers=left,basicstyle=\small,
numberstyle=\tiny,
keywordstyle=\small\color{blue!70}, 
commentstyle=\color{red!50!green!50!blue!50},
frame=shadowbox,
rulesepcolor=\color{red!20!green!20!blue!20}
} 
\lstset{language=C,
    breaklines=true,
    extendedchars=false,
    breakautoindent=true,
    breakindent=10pt,
    numbers=left, 
    basicstyle=\tiny,%\scriptsize,
    numberstyle=\tiny,
    keywordstyle=\tiny\color{blue!70}, 
    commentstyle=\color{red!50!green!50!blue!50},
    escapeinside={(*@}{@*)}, 
  %  columns=cfixed,
    resetmargins=true 
    %  frame=shadowbox,
    %  rulesepcolor=\color{red!20!green!20!blue!20}
}
%\usepackage[CJKbookmarks=true,unicode=true
%bookmarksnumbered=true,
%bookmarksopen=true,
%colorlinks=true,%
%pdfborder=001,
%citecolor=magenta,
%linkcolor=blue,
%linktocpage=true]{hyperref}%超链接, 以及书签
\usepackage[CJKbookmarks=true, bookmarksnumbered=true,
colorlinks=true,% frenchlinks=true, raiselinks=true, xetex,, linktocpage=true 
citecolor=magenta, linkcolor=blue]{hyperref}

\setlength{\parskip}{6pt}


\title{Android MultiMedia}
\author{shecenon@gmail.com}
%\makeindex
\begin{document}
%\begin{CJK}{UTF8}{song}
%\rmfamily
%\sffamily
\ttfamily
\maketitle
% 目录, 标题出现中文,请把.toc中的出错信息删除,或者在标题没有
% 中文时生成.toc文件,然后再在标题中加入中文, 之后就不会出现乱码的错误, 或者在
% 文档的末尾加上\newpage(网上的方法,未经证实)
% 而且要编译两次,甚至三次。
\tableofcontents
\printindex
\newpage
\newcommand{\invoke}[1]{\\ -\textgreater\hspace{#1}}


\part{媒体库}
\section{媒体扫描}
MediaScanner可以通过手动控制,在ANDROID系统中,已经定制了三种事件会触发
MediaScanner去扫描磁盘文件:\newline
ACTION_BOOT_COMPLETED、ACTION_MEDIA_MOUNTED、 ACTION_MEDIA_SCANNER_SCAN_FILE。

其中ACTION_BOOT_COMPLETED是系统启动完后发出这个消息,ACTION_MEDIA_MOUNTED是插
卡事件触发的消息,ACTION_MEDIA_SCANNER_SCAN_FILE消息一般是在一些文件操作后,开
发人员手动发出的一个重新扫描多媒体文件的消息。发送消息通过sendBroadcast函数完
成,比如广播一个ACTION_MEDIA_MOUNTED消息:

\begin{lstlisting}
sendBroadcast(new Intent(Intent.ACTION_MEDIA_MOUNTED, Uri.parse("file://"

                       + Environment.getExternalStorageDirectory())));

                   \end{lstlisting}


frameworks/base/media/java/android/media/MediaScanner.java

packages/providers/MediaProvider/src/com/android/providers/media/MediaScannerService.java

frameworks/base/services/java/com/android/server/MountService.java
notifyVolumeStateChange


frameworks/base/core/java/android/os/Environment.java

frameworks/base/core/java/android/os/storage/IMountService.java

frameworks/base/core/java/android/os/storage/StorageManager.java

frameworks/base/core/java/android/os/storage/IMountService.java

\part{Android Camera Recording}
\section{源码}
% 文档中新增交叉引用后,第一次执行 latex 或 pdflatex 编译命令时
% 会得到类似下面的警告信息。因为第一次编译只会扫描出有交叉引用的地
% 方,第二次编译才能得到正确结果。
\begin{enumerate} 
\setlength{\parskip}{-1pt}
\setlength{\itemsep}{3pt}
\setlength{\parsep}{0ex}
\renewcommand{\labelenumi}{(\theenumi)}
\item \verb|device/samsung/sec_mm/sec_omx/sec_omx_component/video/enc/h264enc/SEC_OMX_H264enc.c|
\item \verb|device/samsung/sec_mm/sec_omx/sec_osal/SEC_OSAL_Log.h|
\item device/sibrary/proprietary/libcamera/SecCamera.cpp
\item device/sibrary/proprietary/libcamera/SecCamera.h
\item \label{sechw}device/sibrary/proprietary/libcamera/SecCameraHWInterface.cpp  
\item frameworks/base/media/libmedia/IMediaRecorder.cpp  
\item frameworks/base/media/libmedia/MediaProfiles.cpp   
\item frameworks/base/media/libmedia/mediarecorder.cpp   
\item frameworks/base/media/jni/android\_media\_MediaProfiles.cpp 
\item frameworks/base/media/jni/android\_media\_MediaRecorder.cpp 
\item \label{camsrc}frameworks/base/media/libstagefright/CameraSource.cpp
\item frameworks/base/media/libstagefright/OMXCodec.cpp
\item frameworks/base/media/libstagefright/OMXCodec.cpp
\item frameworks/base/media/libstagefright/MPEG4Writer.cpp
\item frameworks/base/media/libmediaplayerservice/MediaRecorderClient.cpp  
\item frameworks/base/media/libmediaplayerservice/StagefrightRecorder.cpp  
\item frameworks/base/services/camera/libcameraservice/CameraService.cpp
\item frameworks/base/libs/camera/Camera.cpp             
\item frameworks/base/libs/camera/ICameraClient.cpp      
\item frameworks/base/core/jni/android\_hardware\_Camera.cpp      
\end{enumerate}

\section{Data callback}
\subsection{previewThread thread}
previewThread is the origin region of the recording data callback.
%\begin{lstlisting}[language=C] \end{lstlisting}
\begin{flushleft}
\ref{sechw} SecCameraHWInterface mDataCbTimestamp 
\invoke{10pt} CameraService::Client::dataCallbackTimestamp
\invoke{20pt} handleGenericDataTimestamp
\invoke{30pt} ICameraClient::dataCallbackTimestamp
\invoke{40pt} Through Binder execute : Camera.cpp dataCallbackTimestamp
\invoke{50pt} \ref{camsrc} CameraSource postDataTimestamp
\end{flushleft}

\section{Codecs}
\index{}
\newcommand{\setupVideoEncoder}{setupVideoEncoder()}
\newcommand{\startRecording}{startMPEG4Recording()}
\newcommand{\codeccreate}{ OMXCodec::Create()}
%{~调用~\\\t}%{\Longrightarrow}  will result in "! Missing $ inserted.", because it appear math mode
StagefrightRecorder 模块初始化流程
\begin{flushleft}
    start() \invoke{10pt} \startRecording \invoke{20pt} \setupVideoEncoder
    \invoke{30pt}
\codeccreate
\end{flushleft}
  \setupVideoEncoder 创建了CameraSource \footnote{CameraSource 是 MediaSource 派
  生类}对象并把它传递给\codeccreate,后者把此对象赋值给数据成员
  mSource。\setupVideoEncoder 也把创建的CameraSource对象返回给\startRecording
  ,后者把对象传给MPEG4Writer。
  
 \begin{lstlisting}[language=C] 
\end{lstlisting}



\subsection{sec Init}
    \begin{enumerate} 
\setlength{\parskip}{-1pt}
\item \verb|device/samsung/sec_mm/sec_omx/sec_omx_component/common/SEC_OMX_Basecomponent.c|
\item \verb|device/samsung/proprietary/libstagefrighthw/SEC_OMX_Plugin.cpp|
\end{enumerate}

\begin{flushleft}
\verb|SEC_OMX_Plugin.cpp SECOMXPlugin::SECOMXPlugin()|
\invoke{10pt}\verb|SEC_OMX_Core.c SEC_OMX_GetHandle|
\invoke{20pt}\verb|SEC_OMX_Component_Register.c SEC_OMX_ComponentLoad|
\invoke{30pt}\verb| SEC_OMX_H264enc.c SEC_OMX_ComponentInit |
\invoke{40pt}\verb| SEC_OMX_Venc.c SEC_OMX_VideoEncodeComponentinit |
\invoke{50pt}\verb| SEC_OMX_Basecomponent.c SEC_OMX_BaseComponent_Constructor,|
             \verb| SEC_OMX_Port_Constructor,|
             \verb| SEC_OSAL_Malloc(sizeof(SEC_OMX_VIDEOENC_COMPONENT));|
\end{flushleft}
\begin{verbatim}
SEC_OMX_ComponentInit 是真正初始化的地方,上面的只是加载so库的工作。
SEC_OSAL_dlsym 加载库并找到 SEC_OMX_ComponentInit 函数
SEC_OMX_ComponentLoad的一些工作:
    pOMXComponent = (OMX_COMPONENTTYPE *)SEC_OSAL_Malloc(sizeof(OMX_COMPONENTTYPE));
    INIT_SET_SIZE_VERSION(pOMXComponent, OMX_COMPONENTTYPE);
    ret = (*SEC_OMX_ComponentInit)((OMX_HANDLETYPE)pOMXComponent, sec_component->componentName);


SEC_OMX_Init 只是注册编解码so。

SEC_MFC_H264Enc_Init 调用SsbSipMfcEncInit
\end{verbatim}

\begin{lstlisting}
OSCL_EXPORT_REF OMX_ERRORTYPE SEC_OMX_ComponentInit(OMX_HANDLETYPE hComponent, OMX_STRING componentName)
{
    pOMXComponent = (OMX_COMPONENTTYPE *)hComponent;
    ret = SEC_OMX_VideoEncodeComponentInit(pOMXComponent);
    .. .. ..
    pSECComponent->sec_mfc_componentInit      = &SEC_MFC_H264Enc_Init; // Init MFC
    pSECComponent->sec_mfc_componentTerminate = &SEC_MFC_H264Enc_Terminate;
    pSECComponent->sec_mfc_bufferProcess      = &SEC_MFC_H264Enc_bufferProcess;
    pSECComponent->sec_checkInputFrame        = NULL;
    .. .. ..
}
\end{lstlisting}
\begin{lstlisting}

SEC_OMX_Basecomponent.c
OMX_ERRORTYPE SEC_OMX_ComponentStateSet(OMX_COMPONENTTYPE *pOMXComponent, OMX_U32 messageParam)
{
    .. .. ..
    OMX_STATETYPE          destState = messageParam;
    .. .. .. 
    switch (destState) {
    .. .. .. 
    case OMX_StateIdle:
        switch (currentState) {
        case OMX_StateLoaded:
            for (i = 0; i < pSECComponent->portParam.nPorts; i++) {
                pSECPort = (pSECComponent->pSECPort + i);
                if (CHECK_PORT_TUNNELED(pSECPort) && CHECK_PORT_BUFFER_SUPPLIER(pSECPort)) {
                    if (CHECK_PORT_ENABLED(pSECPort)) {
                        ret = pSECComponent->sec_AllocateTunnelBuffer(pSECPort, i);
                        if (ret!=OMX_ErrorNone)
                            goto EXIT;
                    }
                } else {
                    if (CHECK_PORT_ENABLED(pSECPort)) {
                        SEC_OSAL_SemaphoreWait(pSECComponent->pSECPort[i].loadedResource);
                        pSECPort->portDefinition.bPopulated = OMX_TRUE;
                    }
                }
            }
            ret = pSECComponent->sec_mfc_componentInit(pOMXComponent);
            .. .. ..
        }
        .. .. ..
    }
    .. .. ..
}
\end{lstlisting}
\subsection{I do}
\begin{itemize}
    \item
\begin{lstlisting}
SecCamera fimc_v4l2_dqbuf
int SecCamera::getRecordFrame(void)
{
    int index;
    if (m_flag_record_start == 0) {
        LOGE("%s: m_flag_record_start is 0", __func__);
        return -1;
    }
#ifdef  use_fimc2_rec	
    previewPoll(false);
    index=fimc_v4l2_dqbuf(m_cam_fd2);
    if (0 > index || index >= MAX_BUFFERS) {
        LOGE("ERR(%s):wrong index = %d\n", __func__, index);
        return -1;
    }
    fimc_v4l2_qbuf(m_cam_fd2, index);//add by xusl
    return   index;
#else
#endif
}
\end{lstlisting}
\item so consistence device.mk libstagefrighthw.so 
    \end{itemize}

% 
% 抄录环境(verbatim)在\begin{verbatim}和\end{verbatim} 的任何字符都将原样输出, 包括\等TeX保留字. 
%而"verbatim*"与verbatim的区别是, 它将把空格用|_|表示出来. verbatim有一个简写形式, 
%"\verb标识符 字符串 标识符", 其中标识符可以是空格以外任何字符, 它与\verb之间没有空格.
\begin{verbatim}

\end{verbatim}

%in case of use '\\', if complain that" Underfull \hbox (badness 10000) in paragraph at lines"
%pls check the last '\\' follow a empty line or just like command \end{CJK}
which app to handle file.\\\indent\~{}/.local/share/applications/mimeapps.list  \\
how to enable auto login gconf-editor apps gdm simple-greater disable\_user\_list
%\end{CJK}
\newpage
\end{document}%end fo document
