\documentclass[a4paper,12pt]{article}
\usepackage{verbatim}
%%%%%%%%%%%%%%%%%%%%%%%%%%%%%%%%%%%%%%%%%%%%%%%%%%%%%%%%%%%%%%%%%%%%%%%%%%%%%
%%%%%%%%%%%%%%China font%%%%%%%%%%%%%%%%%%%%
\usepackage{xeCJK}
\setCJKmainfont{AR PL UKai CN}
%\setCJKmainfont{Adobe Kaiti Std}
\setCJKmonofont{文泉驿等宽正黑}
\setCJKsansfont{Ubuntu}%verbatim 所用字体
\punctstyle{banjiao}
%\setCJKfamilyfont{AR PL UKai CN}{kai}
%\setmainfont{Ubuntu}
\usepackage{fontspec}% provides font selecting commands
\usepackage{xunicode}% provides unicode character macros
\usepackage{xltxtra} % provides some fixes/extras
\defaultfontfeatures{Scale=MatchLowercase}
%\setmainfont[Mapping=tex-text]{Baskerville}
%\setsansfont[Mapping=tex-text]{Skia}
%\setmonofont{Courier}

\usepackage{longtable}
\usepackage[CJKbookmarks=true,
bookmarksnumbered=true,
colorlinks=true,
% frenchlinks=true, raiselinks=true, xetex,, linktocpage=true 
citecolor=magenta,
linkcolor=blue]{hyperref}

\usepackage{xcolor}
\usepackage{listings}
\lstset{language=java,
breaklines=true,
extendedchars=false,
breakautoindent=true,
breakindent=10pt,
xleftmargin=30pt, 
numbers=left, 
basicstyle=\scriptsize,
%numberstyle=\tiny,
%keywordstyle=\tiny\color{blue!70}, 
%commentstyle=\color{red!50!green!50!blue!50},
showspaces=false,
showstringspaces=false,
escapeinside={(*@}{@*)} 
%  frame=shadowbox,
%  rulesepcolor=\color{red!20!green!20!blue!20}
}
\usepackage{underscore}

\marginparwidth 100pt
\newcommand{\marginal}[1]{%
\leavevmode\marginpar{\footnotesize\raggedright#1\par}}

\makeatletter
\newcommand{\figcaption}{\def\@captype{figure}\caption}
\newcommand{\tabcaption}{\def\@captype{table}\caption}
\makeatother

%%%%%%%%%%%% watermark %%%%%%%%%%%%%%%%%%
\usepackage{eso-pic}
\usepackage{color}
\makeatletter
  \AddToShipoutPicture{\rm%
    \setlength{\@tempdimb}{.5\paperwidth}%
    \setlength{\@tempdimc}{.5\paperheight}%
    \setlength{\unitlength}{1pt}%
    \put(\strip@pt\@tempdimb,\strip@pt\@tempdimc){%
      \makebox(0,0){\rotatebox{45}{\textcolor[gray]{0.97}%
      {\fontsize{3cm}{3cm}\selectfont{小手网络}}}}
    }
}
\makeatother
\usepackage{fancyhdr}
\pagestyle{fancy}
\addtolength{\headheight}{\baselineskip}
%\renewcommand{\sectionmark}[1]{\markboth{#1}{}}
%\renewcommand{\subsectionmark}[1]{\markright{#1}}
%\rhead{\leftmark\\\rightmark}

\usepackage{graphicx}
\usepackage{multicol}
\usepackage{makeidx}

\renewcommand{\refname}{参考文献}
\renewcommand\contentsname{目录}
\renewcommand\listfigurename{插图目录}
\renewcommand\listtablename{表格目录}
\renewcommand\indexname{索引}
\renewcommand\appendixname{附录}
\renewcommand\figurename{图}
\renewcommand\tablename{表}
\renewcommand{\lstlistlistingname}{代码列表集}
\renewcommand{\lstlistingname}{代码}

\def\equationautorefname{公式}
\def\figureautorefname{图}
\def\subfigureautorefname{子图}
\def\sectionautorefname{节}
\def\subsectionautorefname{子节}
\def\subsubsectionautorefname{小节}
\def\Itemautorefname{项}
\def\tableautorefname{表格}
\def\footnoteautorefname{脚注} 
\def\appendixautorefname{附录}
\author{小手网}
\title{Java 编程规范}
\date{\today}
\makeindex
\begin{document}
\maketitle
\begin{multicols}{2}
\tableofcontents
\lstlistoflistings
\end{multicols}

\tabcaption{修改记录}
\begin{tabular}{|p{0.1\textwidth}|p{0.1\textwidth}|p{0.2\textwidth}
    |p{0.3\textwidth}|p{0.1\textwidth}|}\hline
    日期 & 修订版本 & 修改章节 & 修改描述 & 作者\\\hline
    & v0.1 & & 草稿 & 徐申龙 \\\hline
\end{tabular}

\section{命名规范}
\begin{enumerate} 
    \item 标识符的命名要清晰、明了,有明确含义,同时使用完整的单词或大家基本可以理解的缩写,避免误解。
    \item 除非必要,不要用数字或较奇怪的字符来定义标识符。
    \item 命名中若使用特殊约定或缩写,则要有注释说明。
    \item 含有集合意义的属性命名,尽量包含其复数的意义。
    \item 属性名可以和公有方法参数相同,不能和局部变量相同,引用非静态
        成员变量时使用 this 引用,引用静态成员变量时使用类名引用。
\begin{lstlisting}
public class Person {
    private String name;
    private static List properties;

    public void setName (String name) {
        this.name = name;
    }

    public void setProperties (List properties) {
        Person.properties = properties;
    }
}
\end{lstlisting}
    \item 用正确的反义词组命名具有互斥意义的变量或相反动作的函数等。
        \tabcaption{ 常见的反义词 }
        \begin{tabular}{lll} 
            add / remove    & begin / end      & create / destroy \\
            insert / delete & first / last     & get / release \\
            up / down       & put / get        & cut / paste \\
            add / delete    & lock / unlock    & open / close \\
            min / max       & old / new        & start / stop \\
            next / previous & source / target  & show / hide \\
            send / receive  & source / destination  & increment / decrement
        \end{tabular}
\end{enumerate}

%\begin{longtable}{|p{0.15\textwidth}|p{0.6\textwidth}|p{0.15\textwidth}|}\hline
%\end{longtable}
\subsection{命名的大小写}
\begin{enumerate}
    \item 包名 \hfill\\一般由有3到4个单词组成,全部小写,以"."连接。
        我们项目规定,以com.xiaoshou为前缀,
        以工程名结尾,如 com.xiaoshou.video。
    \item 类,接口 \hfill\\ 由一个或几个单词组成,每个单词
        的第一个字母大写。类一般由名词词组命名,如StringBuffer。
        接口还可以使用形容词,如Runnable, Comparable。
    \item 方法 \hfill\\ 也由一个或几个单词组成,第一个单词全小写,其余的单词首
        字母大写。变量名不应以下划线或美元符号开头,尽管这在语法上是允许的。
        \tabcaption{ 方法命名的一些规则 }
        \begin{tabular}{p{0.25\textwidth}p{0.25\textwidth}p{0.25\textwidth}}\hline
           获取、设置属性值的方法 &形如getXxx(),\newline setXxx(). &
           getPersonInfo(), \newline setPersonInfo.\\\hline
           转换对象类型返回不同类型的方法 &形如 toType &
           toString(),\newline toArray()\\\hline
           返回不同视图的方法 & 形如asType()  & 
           asList(), \newline asGrid()\\\hline
           返回与对象原始类型的值方法 & 形如typeValue() & 
           intValue(),\newline floatValue() \\\hline
           判断状态,返回布尔值 & 形如 isXxx() & isFinish()\\\hline
        \end{tabular}
    \item 域(属性) \hfill\\
        普通域和方法的命名形式一致。 静态的域(也称常量域)由一个或多
        个被下划线分开的单词组成,全部大写。
    \item 局部变量\hfill\\
        命名与域相同,可以使用简写。
\end{enumerate}

\subsection{缩写}
较短的单词可通过去掉“元音”形成缩写;较长的单词可取单词的头几个字母形成缩写;还
有些单词有公认的缩写。
\tabcaption{ 常见的缩写 }
\begin{tabular}{lll} 
    temp      ->  tmp & 
    flag      ->  flg &
    statistic ->  stat \\ 
    increment ->  inc  &
    message   ->  msg  &
    reference ->  ref \\
    pointer   ->  ptr & 
    information -> info &\\
\end{tabular}

如果函数名超过15 个字母,可采用以去掉元音字母的方法或
者以行业内约定俗成的缩写方式缩写函数名。例如,
getCustomerInformation()改为getCustomerInfo()。

\section{排版规范}
\subsection{白空间}
白空间是指空格、退格键(Tab) 以及换行。
记号(包括关键字、变量、常量)与 操作符要适度加入空格。
\begin{enumerate}
    \item 逗号、分号只在后面加空格。
    \item 双目运算符\footnote{双目运算符包括比较运算符:\kern10pt < > >= <= !=
        \hskip15pt 赋值运算符:\kern10pt = -= += /= \%= *= \hskip15pt 
        算术运算符:\kern10pt + - \% *}前后要加空格。
    \item 三元运算符中间要加空格。 
    \item if for while swith 与括号之间加空格。
    \item 单目运算符\footnote{单目运算符包括: ! \~{} ++ -- \& . ::}前后不加空格。
\end{enumerate}
\begin{lstlisting}[language=java,multicols=2,label=space,
    caption=白空间使用示例, showspaces=true,
    showstringspaces=false, numbers=none, frame=shadowbox]
    int a, b, c;
    a = b + c;
    a *= 2;
    a = b ^ 2;
    a++;
    p.id = pid;
\end{lstlisting}
\subsection{缩进}
\begin{enumerate}
    \item 每层的缩进长度为4个空格。
    \item 在编辑器设置中,把退格键设成4个空格的长度。
    \item 类、接口、方法定义开始,循环、判断、分支
        (包括case下的语句)等控制语句等后面换行缩进。
\end{enumerate}
\subsection{换行}
\begin{enumerate}
    \item 包括白空间在内一行不要超过78个ASCII字符。
    \item 长表达式要在低优先级操作符处划分新行,操作符放在新
        行之首,划分出的新行要进行适当的缩进。
    \item 不允许把多个短语句写在一行中,即一行只写一条语句。
        \footnote{一条语句可以跨多行书写。}
    \item 逻辑相对独立的程序块之间、变量说明之后必须加空行。
\end{enumerate}
\section{注释规范}
注释的目的是解释代码的目的、功能和采用的方法,提供
代码以外的信息,帮助读者理解代码,防止没必要的重复
注释信息。

\begin{enumerate}
    \item 源代码文件开始有版权声明、作者等信息,后面要有修改记录。
    \item 在代码的功能、意图层次上进行注释,提供有用、额外的信息。
        注释不宜过多,显而易见的内容就不做注释。
    \item 提供给其他模块的接口,要写明参数、返回值的含义,接口的作用。
    \item 未具体实现的接口/功能,可以用 "//TODO" 标明,实现后要删除。
    \item 提交代码,不要保留大片的注释代码,应删除、精减。
    \item 一般情况下,源程序有效注释量在30%左右为宜。
\end{enumerate}
%\pagecolor{yellow}
\columnseprule.8pt 
\caption{注释示例}
\begin{multicols}{2}
\hfill
如下注释意义不大。
\begin{lstlisting}[numbers=none,basicstyle=\small, backgroundcolor=\color{yellow}] 
// 如果 receiveFlag 为真
if (receiveFlag)
\end{lstlisting}

%    \columnbreak	
\vskip9pt
而如下注释则给出了额外有用的信息 
\begin{lstlisting}[numbers=none,basicstyle=\small,
    backgroundcolor=\color{yellow}]
// 如果从连接收到消息 
if (receiveFlag)
\end{lstlisting}

\vskip9pt
一些复杂的代码需要说明,如下面对闰年算法的说明
\begin{lstlisting}[numbers=none,basicstyle=\scriptsize,
 backgroundcolor=\color{yellow}]
boolean isLeapYear(int year) {
   . . . . . .

/* 1. 如果能被4整除,是闰年;
 * 2. 如果能被100整除,不是闰年;
 * 3. 如果能被400整除,是闰年。
 */
   . . . . . . 

}
 
\end{lstlisting}
\end{multicols}



\section{实现代码规范}
\begin{enumerate} 
    \item 文件要用UTF-8编码, Unix换行格式。不要使用cp936、
        gb1232、gb18030编码,Dos风格的换行。
    \item 方法体不要太长,尽量不超过200行, 60到70行是比较合适的。
    \item 如果多段代码重复做同一件事情,那么在方法的划分上可能存在问题。
    \item 关系较为紧密的代码应尽可能相邻。
    \item 应明确规定对接口方法参数的合法性检查应由方法
        的调用者负责还是由接口方法本身负责,缺省是由方法调用者负责。
        \footnote {对于模块间接口方法的参数的合法性检查这一问题,
        往往有两个极端现象,即:要么是调用者和被调用者对参数均不
        作合法性检查,结果就遗漏了合法性检查这一必要的处理过程,
        造成问题隐患;要么就是调用者和被调用者均对参数进行合法性
        检查,这种情况虽不会造成问题,但产生了冗余代码,降低了效率。}
    \item 数据库的设计要尽量考虑向前兼容和以后的版本升级,
        并为某些未来可能的应用保留余地(如预留一些属性等)。
\end{enumerate}

\subsection{语法的注意实现}
\begin{enumerate} 
    \item 编码要有防错处理,如对空指针要做判断,避免异常的扩散。
        可能有产生异常的代码,放到try\ldots catch。
    \item 避免使用不易理解的数字,用有意义的标识来替代。涉及物理状态或
        者含有物理意义的常量,不应直接使用数字,必须用有意义的静态变量来代替。
    \item 注意运算符的优先级,并用括号明确表达式的操作顺序,避免使用默认
        优先级。
    \item 明确方法功能,精确(而不是近似)地实现方法设计。
        一个函数仅完成一件功能,即使简单功能也应该编写方法实现。
    \item 明确类的功能,精确(而非近似)地实现类的设计。一个类仅实现一组相近的功能。
        划分类的时候,应该尽量把逻辑处理、数据和显示分离,实现类功能的单一性。
    \item 所有的数据类必须重载toString() 方法,返回该类有意义的内容。
        父类如果实现了比较合理的toString() ,子类可以继承不必再重写。
\end{enumerate}


\subsection{避免内存泄漏}
\begin{enumerate}
    \item 数据库游标、文件I/O操作等需要使用结束close()的对象必须在
        try-catch-finally 的finally中close()。
    \item 集合中的数据如果不使用了应该及时释放,尤其是可重复使用的集合。
        \\由于集合保存了对象的句柄,虚拟机的垃圾收集器就不会回收。 
    \item 代码中注册的ContentObserver,BroadcastReceiver 对象在 
        Activity 销毁前要取消注册。
    \item Bitmap 对象在使用后,要及时回收。
\end{enumerate}
\begin{lstlisting}[language=java,basicstyle=\scriptsize,caption=内存泄漏示例]
public class Blackboard extends Activity {
    private final static String TAG = "MemoryLeakTest";

    private static Drawable mBackground;
    static List<byte[]> mRawData = new ArrayList<byte[]>();

    public void onCreate(Bundle savedInstanceState) {
        super.onCreate(savedInstanceState);        

        TextView label = new TextView(this);
        mBackground = getResources().getDrawable(R.drawable.greatwall);
        Bitmap mBitmap = BitmapFactory.decodeResource(getResources(), 
                                        R.drawable.greatwall); 
        label.setBackgroundDrawable(mBackground); 
        setContentView(label); 

        //staticCollectionLeak();
        //fileLeakTest();
        //cursorLeakTest();
        //registerLeakTest();
        contentObserverLeakTest();
    }

    /*
     * 测试结果: 造成 dalvik 的内存泄漏。 hprof 可以定位。
     */
    private void staticCollectionLeak() {
        byte[] mass = new byte[8 * 1024 * 1024];
        mRawData.add(mass);
    }

    /*
     * IO 未 close. 
     */
    private void fileLeakTest( ) {
        try{
            FileOutputStream fos;
            int i = 0;
            String contents = "Write some test to file\n";
            fos = openFileOutput("fileLeakTest.txt", MODE_PRIVATE);
            while (i++ < 1000) {
                fos.write(contents.getBytes());
            }
        } catch (Exception e) {
            Log.e(TAG, "oops, catch Exception " + e);
        } finally {
            //fos.close();
        }
    }

    /*
     * 测试结果: IntentFilter 未 unregisterReceiver 会造成native
     * 内存泄漏。 但当进程重启时会报错:
     *    Activity com.xiaoshou.classroom.Blackboard has leaked
     *    IntentReceiver com.xiaoshou.classroom.Blackboard$1@40525d08
     *    that was originally registered here. Are you missing a call
     *    to unregisterReceiver()?
     */
    private void registerLeakTest() {
        IntentFilter filter = new IntentFilter();
        filter.addAction(Intent.ACTION_DATE_CHANGED);
        filter.addAction(Intent.ACTION_BATTERY_CHANGED);
        registerReceiver(mIntentReceiver, filter);
    }

    /*
     * 测试结果: 未unregisterContentObserver 时会造成 native 内存泄漏。
     */
    private void contentObserverLeakTest() {
        getContentResolver().registerContentObserver(
                Settings.System.CONTENT_URI, true, mDateFomat);
    }

    /*
     * 测试结果: Cursor 未 close 时会有如下警告,有时要多测几次:
     *    StrictMode  E  Finalizing a Cursor that has not been 
     *    deactivated or closed. database = 
     *    /data/data/com.android.soundrecorder/databases/simple,
     *    table = soundrecorder, query = SELECT * FROM soundrecorder
     *    ORDER BY _id
     */ 
    private void cursorLeakTest() {
        String[] projection = new String[] {Media._ID, Media.TITLE};
        Cursor cursorImage = managedQuery(Media.EXTERNAL_CONTENT_URI, 
                      projection, null, null, Media.DATE_TAKEN + " ASC");
    }

    private final ContentObserver mDateFomat = 
        new ContentObserver(new Handler()) {
            public void onChange(boolean selfChange) {
            }
    };

    private final BroadcastReceiver mIntentReceiver = 
                                        new BroadcastReceiver() {
        public void onReceive(Context context, Intent intent) {
            final String action = intent.getAction();
            if (Intent.ACTION_DATE_CHANGED.equals(action)) {
            } else if (Intent.ACTION_BATTERY_CHANGED.equals(action)) {
            }
        }
    };
}
\end{lstlisting}

\subsection{异常处理}
\begin{enumerate}
    \item 异常捕获后,如果不对该异常进行处理,则应该输出
        日志或者调用栈Exception.printStackTrace()。若有
        特殊原因必须用注释加以说明。
    \item 一个方法不应抛出太多类型的异常。
    \item 异常捕获尽量不要直接 catch (Exception ex),应该把异常细分处理。
    \item 在程序中使用异常处理还是使用错误返回码处理,根据是否有利于程
        序结构来确定,并且异常和错误码不应该混合使用,推荐使用异常。
        \footnote{一个系统或者模块应该统一规划异常类型和返回码的含义。 
        但是不能用异常来做一般流程处理的方式,不要过多地使用异常,异常的处理效
        率比条件分支低,而且异常的跳转流程难以预测。}
\end{enumerate}

\subsection{日志}
\begin{enumerate}
    \item 在异常处理、防错处理的代码块,要有日志。
    \item 增加日志调试开关,在发行版本关闭调试,开发的调试版本打开。
    \item 在提交代码前,删除不需要的调试 log,避免过多的日志。
    \item 尽量避免使用 System.out.println,首选android.util.Log,比如 Log.d, Log.e, Log.w
\end{enumerate}
%\medbreak
%\vfill\eject
%\appendix
%\addcontentsline{toc}{section}{Appendices}\markboth{APPENDICES}{}
%\section{代码审查表}
\end{document}
