\documentclass[a4paper,11pt,]{article}% a4paper - A4纸 11pt -字体 twoside -双面 openany -新章节可在偶数页开始
%\usepackage{CJKutf8}%Chinese layout
\usepackage{xeCJK}
\setCJKmainfont{FangSong}
\setCJKmonofont{Adobe Kaiti Std}
\punctstyle{banjiao}
%\setCJKfamilyfont{AR PL UKai CN}{kai}
\setmainfont{DejaVu Sans}
\pagestyle{empty} %不显示页码
%\input{pagegeometry}%自定义的文件,设置页面参数
%\usepackage{amsmath,amsthm,amsfonts,amssymb,bm} % 数学宏包
%\usepackage{graphicx,psfrag}                    % 图形宏包
\usepackage{makeidx}                            % 建立索引宏包
\usepackage[titletoc]{appendix}
\usepackage{verbatim}
\usepackage{comment}
\usepackage{mdwlist}
\usepackage{underscore}%正文使用_
\usepackage{ifthen}
\usepackage{xcolor}
\usepackage{tikz}
\usetikzlibrary{arrows,decorations.pathmorphing,positioning,backgrounds,chains,fit,calc,petri,through,shapes}
\usepackage[shell,tikz]{dot2texi}
\usepackage[CJKbookmarks=true, bookmarksnumbered=true,
colorlinks=true,% frenchlinks=true, raiselinks=true, xetex,, linktocpage=true 
citecolor=magenta, linkcolor=blue]{hyperref}
\usepackage{listings}%源代码宏包,such as C/C++ code
\lstset{language=C,
    breaklines=true,
    extendedchars=false,
    breakautoindent=true,
    breakindent=10pt,
    numbers=left, 
    basicstyle=\tiny,%\scriptsize,
    numberstyle=\tiny,
    keywordstyle=\tiny\color{blue!70}, 
    commentstyle=\color{red!50!green!50!blue!50},
    escapeinside={(*@}{@*)}, 
    %  columns=cfixed,
    resetmargins=true 
    %  frame=shadowbox,
    %  rulesepcolor=\color{red!20!green!20!blue!20}
}
\setlength{\parskip}{6pt}

\title{Android Wifi}
\author{shecenon@gmail.com}

\begin{document}
%\rmfamily
%\sffamily
\ttfamily
\maketitle
% 目录, 标题出现中文,请把.toc中的出错信息删除,或者在标题没有
% 中文时生成.toc文件,然后再在标题中加入中文, 之后就不会出现乱码的错误, 或者在
% 文档的末尾加上\newpage(网上的方法,未经证实)
% 而且要编译两次,甚至三次。
\tableofcontents
\printindex
\newpage
\renewcommand{\refname}{参考文献}
\renewcommand\contentsname{目录}
\renewcommand\listfigurename{插图目录}
\renewcommand\listtablename{表格目录}
\renewcommand\indexname{索引}
\renewcommand\appendixname{附录}
\renewcommand\figurename{图}
\renewcommand\tablename{表}
\renewcommand{\lstlistlistingname}{代码列表集}
\renewcommand{\lstlistingname}{代码}

\def\equationautorefname{公式}%
\def\figureautorefname{图}%
\def\subfigureautorefname{子图}%
\def\sectionautorefname{节}%
\def\subsectionautorefname{子节}%
\def\subsubsectionautorefname{小节}%
\def\Itemautorefname{项}%
\def\tableautorefname{表格}%
\def\footnoteautorefname{脚注} 
\def\appendixautorefname{附录}
\newcommand{\invoke}[1]{\\ -\textgreater\hspace{#1}}
\newcommand{\methodnode}[3]{\rnode{#3}{\psframebox{#1#2\ref{#1}~#3}}\vspace{20pt}\newline }
\newcommand{\methodcall}[2]{\ncarc{->}{#1}{#2}\ncput*{call}}
%\tiny
\small
\part{Android Wifi}
\section{概述wpa_supplicant}
\begin{verbatim}
wpa_supplicant module actually provider  wpa_supplicant and libwpa_client, the
former is a executable program, act as wpa service who defined by init.rc, the
latter is a share library used by HAL, that is libhardware_legacy.
libwpa_client only contains os_unix.c and wpa_ctrl.c,
attention that wpa_ctrl.c only exist in libwpa_client, not in wpa_supplicant.
\end{verbatim}

\subsection{WifiMonitor.MonitorThread}
previewThread is the origin region of the recording data callback.
%\begin{figure}
%\begin{center} \end{center}
%\psset{nodesep=6pt}
%\methodnode{WifiStateTracker}{.java}{startEventLoop}
%\methodnode{WifiMonitor}{.java}{startMonitoring}
%\methodnode{WifiMonitor}{.java}{MonitorThread}
%\methodcall{startEventLoop}{startMonitoring}
%\methodcall{startMonitoring}{MonitorThread}
%\end{figure}
run()
handleEvent
\index{}
\begin{enumerate}  
    \item{start wifi}

        WifiService.setWifiEnabledBlocking( ) 

        mWifiStateTracker.loadDriver()  MESSAGE_ENABLE_WIFI 

    \item{scan }

        startScan MESSAGE_START_SCAN

    \item{connect AP}

        WifiService.enableNetwork()

        mWifiStateTracker.enableNetwork(netId, disableOthers);
\end{enumerate}

\section{初始化}
\begin{dot2tex}[neato,autosize]
graph G {
    node [shape=rectangle];
    main -> {init; addiface; run}[style="arrows=->"];
    addiface -> {initiface; initiface2}[style="arrows=->"];
    init -> {eaps; eloopinit; wsgci}[style="arrows=->"];
    run -> elooprun[style="arrows=->"];

    run [label="wpa_supplicant_run"];
    init [label="wpa_supplicant_init"];
    addiface [label="wpa_supplicant_add_iface"];
    elooprun [label="eloop_run"];
    eaps[label="eap_peer_register_methods"];
    eloopinit[label="eloop_init"];
    wsgci[label="wpa_supplicant_global_ctrl_iface_init"];
    initiface[label="wpa_supplicant_init_iface"];
    initiface2[label="wpa_supplicant_init_iface2"];
}
\end{dot2tex}

\begin{dot2tex}[dot,autosize]
graph G {
    node [shape=rectangle];
    initiface -- {setdrv; config}[style="arrows=->"];
    setdrv -- drivers[style="arrows=->"];

    initiface[label="wpa_supplicant_init_iface"];
    setdrv[label="wpa_supplicant_set_driver"];
    config[label="wpa_config_read"];
    drivers[label="wpa_supplicant_drivers"];
}
\end{dot2tex}



\begin{dot2tex}[neato,autosize]
graph G {
    node [shape=rectangle];
    initiface2 -> {initeapol; drvinit; initwpa; driverinit; ctrlifaceinit; initiface}[style="arrows=->"];
    initwpa -> {initsm}[style="arrows=->"];
    driverinit -> {initl2}[style="arrows=->"];

    initiface2[label="wpa_supplicant_init_iface2"];
    initeapol [label="wpa_supplicant_init_eapol"];
    drvinit [label="wpa_drv_init"];
    initwpa [label="wpa_supplicant_init_wpa"];
    driverinit [label="wpa_supplicant_driver_init"];
    ctrlifaceinit [label="wpa_supplicant_ctrl_iface_init"];
    initiface[label="ieee80211_sta_init"];

    initsm [label="wpa_sm_init"];
    initl2 [label="l2_packet_init"];
}
\end{dot2tex}

\begin{lstlisting}[language=C,caption=wpa_driver_wext_init] 
    struct wpa_driver_wext_data (*@\underline{ *drv;}@*)
    drv->ioctl_sock = socket(PF_INET, SOCK_DGRAM, 0);
    . . .
    s = socket(PF_NETLINK, SOCK_RAW, NETLINK_ROUTE);
    . . .
    eloop_register_read_sock(s, wpa_driver_wext_event_receive, drv, ctx);
    drv->event_sock = s;
\end{lstlisting}

\normalsize
驱动和EAP的注册是类似的,:
wpa_supplicant_add_iface -> wpa_supplicant_init_iface -> wpa_supplicant_set_driver ->
drivers.c  wpa_supplicant_drivers -> driver_wext.c wpa_driver_wext_ops -> ioctl

eap_methods.c eap_peer_register_methods  -> eap_psk.c eap_peer_psk_register

wpa_supplicant_add_iface -> wpa_supplicant_init_iface2 ->wpa_supplicant_init_eapol, wpa_drv_init,
wpa_drv_get_ifname, wpa_supplicant_init_wpa, wpa_supplicant_driver_init, wpa_supplicant_ctrl_iface_init
ieee80211_sta_init

wpa_drv_init  与  wpa_supplicant_driver_init 的区别:

wpa_drv_init 是调用了struct wpa_supplicant里的 driver->init, 
即 driver_wext 的 wpa_driver_wext_init。
这个包裹函数在 wpa_supplicant_i.h 定义的,类似的还有 wpa_drv_deinit,
wpa_drv_set_param, wpa_drv_set_wpa,wpa_drv_scan, wpa_drv_associate,
wpa_drv_get_bssid 等等。

wpa_supplicant_driver_init 设置接口的参数,所以她必须在wpa_drv_init 之后调用。
初始化并设置了驱动的参数后,便调用 wpa_supplicant_req_scan 扫描热点了。
wpa_supplicant_req_scan 调用eloop_register_timeout 把函数 wpa_supplicant_scan
、数据 struct wpa_supplicant 注册到eloop中\label{regscan}。
调用wpa_supplicant_req_scan的地方还有:
wpa_supplicant_scan
events 模块
ctrl_iface 模块
%fixme
events 模块的函数 wpa_supplicant_event ,这个是路由功能性函数,就像 biner 的
transact/onTransact。 这个函数就是被 driver_wext 调用的, 除此之外
wpa_supplicant 模块的 wpa_supplicant_gen_assoc_event,  
events 模块会调用 sm 模块(wpa.c),所以,驱动模块会通知 sm 模块的。

\section{事件循环}
eloop  就是 event loop。 其核心就是 select 。客户调用eloop的注册函数,
把socket 、 回调函数、 以及上下文相关的数据告诉 eloop。 eloop 在 select 里等待
,收到消息后,调用注册的回调函数。

最重要的数据是:\\
static struct eloop_data eloop;

从外部来看,主要是eloop_register_timeout
\scriptsize\newline
\begin{tabular}{|p{0.4\textwidth}|p{0.36\textwidth}|p{0.22\textwidth}|}\hline
    调用者 & 注册的句柄 & 所在源文件\\\hline
    wpa_driver_wext_scan  & wpa_driver_wext_scan_timeout& driver_wext.c\\\hline
    wpa_supplicant_event_michael_mic_failure & wpa_supplicant_stop_countermeasures&events.c \\\hline
    wpa_supplicant_key_neg_complete & wpa_sm_start_preauth & wpa.c\\\hline
    wpa_supplicant_process_smk_m45 &wpa_supplicant_smk_timeout & wpa.c \\\hline
    wpa_supplicant_update_smk_lifetime &wpa_supplicant_smk_timeout & wpa.c\\\hline
    eapol_port_timers_tick & eapol_port_timers_tick & eapol_sm.c\\\hline
    eapol_sm_step & eapol_sm_step_timeout & eapol_sm.c\\\hline
    eapol_sm_init & eapol_port_timers_tick & eapol_sm.c\\\hline
    pmksa_cache_set_expiration & pmksa_cache_expire&pmksa_cache.c \\\hline
    pmksa_cache_set_expiration & pmksa_cache_reauth&pmksa_cache.c \\\hline
    rsn_preauth_init & rsn_preauth_timeout & preauth.c \\\hline
    wpa_supplicant_req_scan &wpa_supplicant_scan &wpa_supplicant.c \\\hline
    wpa_supplicant_req_auth_timeout &wpa_supplicant_timeout &wpa_supplicant.c \\\hline
\end{tabular}
\normalsize


eloop_register_read_sock 调用的地方:
\scriptsize\newline
\begin{tabular}{|p{0.36\textwidth}|p{0.4\textwidth}|p{0.22\textwidth}|}\hline
    调用者 & 注册的句柄 & 所在源文件\\\hline
    wpa_driver_wext_init & wpa_driver_wext_event_receive & driver_wext.c \\\hline
    wpa_driver_wext_open_mlme & wpa_driver_wext_mlme_read &driver_wext.c \\\hline
    wpa_supplicant_ctrl_iface_init & wpa_supplicant_ctrl_iface_receive &ctrl_iface_unix.c \\\hline
    wpa_supplicant_global_ctrl_iface_init & wpa_supplicant_global_ctrl_iface_receive & ctrl_iface_unix.c \\\hline
    l2_packet_init & l2_packet_receive & l2_packet_linux.c \\\hline
\end{tabular}
\normalsize
eloop_register_read_sock 调用了 eloop_register_sock.


wpa_supplicant_run 调用的两个注册函数:

\begin{itemize*}
    \item eloop_register_signal_terminate
    \item eloop_register_signal_reconfig
\end{itemize*}
wpa_supplicant_run 是这两个函数唯一被调用的地方。 
注册的句柄分别为:wpa_supplicant_terminate, wpa_supplicant_reconfig
而 eloop_register_signal 主要别这两个函数调用。



l2_packet_init
l2_packet_send 被 wap_supplicant 模块的函数 wpa_ether_send 、
preauth 模块的函数 rsn_preauth_eapol_send 。由于这两个函数都是静态函数,
前者被wpa_supplicant_eapol_send、 _wpa_ether_send
后者赋给了struct eapol_ctx 的指针 eapol_send

rsn preauth 初始化流程:
wpa_supplicant_ctrl_iface_receive (ctrl_iface_unix.c) 调用
wpa_supplicant_ctrl_iface_process (ctrl_iface.c) 调用 wpa_supplicant_ctrl_iface_preauth 调用 rsn_preauth_init



\subsection{套接字}
socket
driver_wext.c  wpa_driver_wext_init
ctrl_iface_unix.c
l2_packet_linux.c


PF_NETLINK 就是 driver_wext.c 模块使用了。

connect
\begin{lstlisting}[language=,]
./ctrl_iface_unix.c:352:		if (connect(priv->sock, (struct sockaddr *) &addr,
./ctrl_iface_unix.c:651:		if (connect(priv->sock, (struct sockaddr *) &addr,

./wpa_ctrl.c:129:	if (connect(ctrl->s, (struct sockaddr *) &ctrl->dest,
./wpa_ctrl.c:231:	if (connect(ctrl->s, (struct sockaddr *) &ctrl->dest,
\end{lstlisting}


Netlink的RTMGRP_LINK多播群组,用户空间程序可以注册netlink的RTMGRP_LINK多播群组,
当设备的状态或配置中有变更时,就会用rtmsg_ifinfo把通知信息传送给Link多播群组RTMGRP_LINK,
其中一些通知信息如下:
 
当netdev_chain通知链收到一个通知信息时,RTnetlink会注册的netdev_chain,
然后重放其接受到的通知信息。
当一个已关闭的设备开启时或者处于相反的过程。
当net_device->flags中的一个标识有变动时。
netplugd是守护进程,会监听这些通知信息,然后根据用户配置文件而反应。

\begin{lstlisting}[language=,]

./include/linux/rtnetlink.h:520:#define RTMGRP_LINK		1

android_net_ethernet_initEthernetNative
frameworks/base/core/jni/android_net_ethernet.cpp:319:        addr_poll.nl_groups = RTMGRP_LINK | RTMGRP_IPV4_IFADDR;

\end{lstlisting}

\begin{lstlisting}{language=,}
net/wireless/wext-core.c:	nlh = nlmsg_put(skb, 0, 0, RTM_NEWLINK, sizeof(*r), 0);
net/bridge/br_if.c:	br_ifinfo_notify(RTM_NEWLINK, p);
net/bridge/br_netlink.c:	if (event == RTM_NEWLINK)
net/bridge/br_netlink.c:				   cb->nlh->nlmsg_seq, RTM_NEWLINK,
net/bridge/br_notify.c:		br_ifinfo_notify(RTM_NEWLINK, p);
net/core/rtnetlink.c:	[RTM_FAM(RTM_NEWLINK)]      = NLMSG_LENGTH(sizeof(struct ifinfomsg)),
net/core/rtnetlink.c:	[RTM_FAM(RTM_NEWLINK)]      = IFLA_MAX,
net/core/rtnetlink.c:			if (rtnl_fill_ifinfo(skb, dev, RTM_NEWLINK,
net/core/rtnetlink.c:	rtmsg_ifinfo(RTM_NEWLINK, dev, ~0U);
net/core/rtnetlink.c:	err = rtnl_fill_ifinfo(nskb, dev, RTM_NEWLINK, NETLINK_CB(skb).pid,
net/core/rtnetlink.c:		rtmsg_ifinfo(RTM_NEWLINK, dev, 0);
net/core/rtnetlink.c:	rtnl_register(PF_UNSPEC, RTM_NEWLINK, rtnl_newlink, NULL);
net/core/dev.c:		rtmsg_ifinfo(RTM_NEWLINK, dev, 0);
net/core/dev.c:	rtmsg_ifinfo(RTM_NEWLINK, dev, IFF_UP|IFF_RUNNING);
net/core/dev.c:	rtmsg_ifinfo(RTM_NEWLINK, dev, IFF_UP|IFF_RUNNING);
net/core/dev.c: *	are adjusted, %RTM_NEWLINK is sent to the routing socket and the
net/core/dev.c:	rtmsg_ifinfo(RTM_NEWLINK, slave, IFF_SLAVE);
net/core/dev.c:		rtmsg_ifinfo(RTM_NEWLINK, dev, changes);
net/core/dev.c:		rtmsg_ifinfo(RTM_NEWLINK, dev, ~0U);
net/core/dev.c:	rtmsg_ifinfo(RTM_NEWLINK, dev, ~0U);



include/linux/wireless.h: * space. They are encapsulated in the IFLA_WIRELESS field of
include/linux/if_link.h:	IFLA_WIRELESS,		/* Wireless Extension event - see wireless.h */
include/linux/if_link.h:#define IFLA_WIRELESS IFLA_WIRELESS
net/wireless/wext-core.c:	nla = nla_reserve(skb, IFLA_WIRELESS, event_len);
net/wireless/wext-core.c:	nla = nla_reserve(compskb, IFLA_WIRELESS, event_len);
\end{lstlisting}

include/linux/netlink.h
inculde/linux/rtnetlink.h

driver_wext.c
wpa_driver_wext_event_receive   RTM_NEWLINK
wpa_driver_wext_event_rtm_newlink IFLA_WIRELESS , IFLA_IFNAME

wpa_driver_wext_event_wireless  在处理 SIOCGIWSCAN ,以及
wpa_driver_wext_event_link 
都调用了
wpa_supplicant_event

总结一下, wpa_driver_wext_event_receive 是 wpa_driver_wext_init 向 eloop 的 eloop_register_read_sock
注册的回调函数。
所以,情况是eloop 的select 等到了 socket 有数据了,调用 wext_driver 的回调接口
,回调接口最后发给了 wpa_supplicant_event.


wpa.c 是 wpa 的状态机,
eapol_sm.c 是 eapol 的状态机。
eapol_sm_step  最后的 sm->ctx->cb(sm, success, sm->ctx->cb_ctx);
实际调用的是 rsn_preauth_eapol_cb





wext-core.c
wireless_send_event 调用 rtnetlink_ifinfo_prep
\begin{lstlisting}
static struct nlmsghdr *rtnetlink_ifinfo_prep(struct net_device *dev,
					      struct sk_buff *skb)
{
	struct ifinfomsg *r;
	struct nlmsghdr  *nlh;

	nlh = nlmsg_put(skb, 0, 0, RTM_NEWLINK, sizeof(*r), 0);
	if (!nlh)
		return NULL;

	r = nlmsg_data(nlh);
	r->ifi_family = AF_UNSPEC;
	r->__ifi_pad = 0;
	r->ifi_type = dev->type;
	r->ifi_index = dev->ifindex;
	r->ifi_flags = dev_get_flags(dev);
	r->ifi_change = 0;	/* Wireless changes don't affect those flags */

	NLA_PUT_STRING(skb, IFLA_IFNAME, dev->name);

	return nlh;
 nla_put_failure:
	nlmsg_cancel(skb, nlh);
	return NULL;
}
\end{lstlisting}

%\begin{dot2tex}[shell,dot]
%    graph struct{
%    node [shape=record];
%        structs [shape=record, label="<f0> |<f1> |<f2> "];
%        structd [shape=record, label="<f0> |<f1> |<f2> "];
%        structd -> structs;
%        }
%        
%\end{dot2tex}

\section{接口打开}
\newcommand{\wpactrl}{wpa\_ctrl.c}

WifiManager is a client of WifiService.

WifiStateTracker is a object class of WifiService, meanwhile WifiStateTracker
hold WifiService object. 

The relationship between WifiStateTracker and
WifiMonitor is similar to WifiService and WifiStateTracker.

WifiNative of provider wap supplicant for WifiStateTracker. It through jni
\verb|android_net_wifi_Wifi.cpp|  call wifi.c, and wifi.c call wap_supplicant
through wpa_ctrl.c, actually wifi.c is the only caller of wpa_ctrl.c.

wpa_ctrl.c define wpa and HAL socket communicate path , 
\begin{lstlisting}[language=C]
static const char *local_socket_dir = "/data/misc/wifi/sockets";
static const char *local_socket_prefix = "wpa_ctrl_";
\end{lstlisting}

if system_server pid is 79, then there are two socket that bind to the
following path:
 \begin{lstlisting}[language=C]
/data/misc/wifi/sockets/wpa_ctrl_79-0
/data/misc/wifi/sockets/wpa_ctrl_79-1
\end{lstlisting}
the first is ctrl socket, the last is monitor socket.  wifi.c
wifi_connect_to_supplicant() do create them by call wpa_ctrl_open("wlan0").
wpa_ctrl_open also connet these two sockets to /dev/socket/wpa_wlan0 through
invoke socket_local_client_connect method.

 \begin{lstlisting}[language=,] 
system/core/libcutils/socket_local_client.c
system/core/libcutils/socket_local.h:#define ANDROID_RESERVED_SOCKET_PREFIX "/dev/socket/"


service wpa_supplicant /system/bin/wpa_supplicant -Dwext -iwlan0 -c/data/misc/wifi/wpa_supplicant.conf -dd
    user root
    group system wifi inet 
    socket wpa_wlan0 dgram 660 wifi wifi
    disabled
    oneshot
\end{lstlisting}
The statement `socket wpa_wlan0 dgram 660 wifi wifi` indicate init to create socket /dev/socket/wpa_wlan0
(this work is done by create_socket in util.c). And init will create
environment `ANDROID_SOCKET_wlan0`(ANDROID_SOCKET_ENV_PREFIX + `wlan0` ) to achive publish_socket()
\begin{lstlisting}[language=,] 
system/core/include/cutils/sockets.h:#define ANDROID_SOCKET_DIR		"/dev/socket"
\end{lstlisting} 

\begin{lstlisting}[language=bash, caption=init.rc]
    mkdir /system/etc/wifi 0770 wifi wifi
    chmod 0770 /system/etc/wifi
    chmod 0660 /system/etc/wifi/wpa_supplicant.conf
    chown wifi wifi /system/etc/wifi/wpa_supplicant.conf
    #wpa_supplicant control socket for android wifi.c (android private socket)
    mkdir /data/misc/wifi 0770 wifi wifi
    mkdir /data/misc/wifi/sockets 0770 wifi wifi
    chmod 0770 /data/misc/wifi
    chmod 0660 /data/misc/wifi/wpa_supplicant.conf
    chown wifi wifi /data/misc/wifi
    chown wifi wifi /data/misc/wifi/wpa_supplicant.conf
    setprop wifi.interface "wlan0"
    #setprop wlan.driver.status "ok"

    service wpa_supplicant /system/bin/wpa_supplicant -Dwext -iwlan0 -c/data/misc/wifi/wpa_supplicant.conf -dd
        user root
        group wifi inet 
        socket wpa_wlan0 dgram 660 wifi wifi   (*@\label{socketwpa}@*) 
        disabled
        oneshot

    service dhcpcd /system/bin/dhcpcd -BKL wlan0
        disabled
        oneshot 
\end{lstlisting}

\begin{lstlisting}[language=C,caption=wpa_supplicant_ctrl_iface_init]
#ifdef ANDROID
	os_snprintf(addr.sun_path, sizeof(addr.sun_path), "wpa_%s",
			wpa_s->conf->ctrl_interface);
	priv->sock = android_get_control_socket(addr.sun_path);
	if (priv->sock >= 0)
		goto havesock;
#endif
	if (os_strncmp(buf, "DIR=", 4) == 0) {
		dir = buf + 4;
		gid_str = os_strstr(dir, " GROUP=");
		if (gid_str) {
			*gid_str = '\0';
			gid_str += 7;
		}
	} else {
		dir = buf;
		gid_str = wpa_s->conf->ctrl_interface_group;
	}

         . . .
havesock:
	eloop_register_read_sock(priv->sock, wpa_supplicant_ctrl_iface_receive,
				 wpa_s, priv);
	wpa_msg_register_cb(wpa_supplicant_ctrl_iface_msg_cb);

\end{lstlisting}
android_get_control_socket is through environment `ANDROID_SOCKET_wlan0` to
get socket fd. This socket is created in init.rc \autoref{socketwpa} named  /dev/socket/wpa_wlan0
.

 在 ctrl_iface_unix.c 中,
wpa_supplicant_global_ctrl_iface_init 为套接字注册了回调  wpa_supplicant_global_ctrl_iface_receive
,  wpa_supplicant_ctrl_iface_init 为套接字注册了回调 wpa_supplicant_ctrl_iface_receive
他们所以的套接字是unix 域套接字,但是是 Android 作了定制的。
socket(PF_UNIX, SOCK_DGRAM, 0);
addr.sun_family = AF_UNIX;

此套接字是 init.rc 定义的由 init 进程创建的,名叫 wpa_wlan0 \autoref{socketwpa}。

wpa_ctrl.c wpa_ctrl_open 也会链接到这个套接字上的。
wpa_ctrl_attach 发出的指令通过 sock,经wpa_supplicant_ctrl_iface_receive 转发
给wpa_supplicant_ctrl_iface_attach。

而 wpa_ctrl.c 是给 wpa_cli 命令 和 wifi HAL (即 wifi.c)调用的源码。

wpa_supplicant_ctrl_iface_receive 把大量的处理交给ctrl_iface.c 的函数
wpa_supplicant_ctrl_iface_process 来处里。
这个process 函数处理的绝大数命令是 wpa_cli.c 里发出的命令。
所以,这是通过 socket 的进程间通讯。
\section{热点扫描}

wpa_supplicant_req_scan 把 wpa_supplicant_scan 注册到 eloop。
这是一个timeout 事件。

wpa_supplicant_scan\label{scan} 里有两天路径

1. 直接调用wpa_supplicant_event(wpa_s, EVENT_SCAN_RESULTS, NULL);

2.	ret = wpa_drv_scan(wpa_s, ssid ? ssid->ssid : NULL,\\
				   ssid ? ssid->ssid_len : 0); 
这个对应到 wpa_driver_wext_scan,此函数会向 eloop 注册 wpa_driver_wext_scan_timeout
, 这个超时回调也调用了wpa_supplicant_event(timeout_ctx, EVENT_SCAN_RESULTS, NULL);

wpa_supplicant_event 会调用 wpa_supplicant_event_scan_results 处理扫描结果。
 wpa_supplicant_event_scan_results 首先调用了 wpa_supplicant_get_scan_results
后者调用wpa_drv_get_scan_results,这个是对 get_scan_results 方法的包装, 所以
,会走到 wpa_driver_wext_get_scan_results。
接着,调用 wpa_supplicant_associate.  
然后,调用rsn_preauth_scan_results

wpa_supplicant_req_scan
被 wpa_supplicant_driver_init,
wap_supplicant_timeout,
wpa_supplicant_reloading_configuration,
wap_supplicant_disassociate,
wap_supplicant_deauthticate
以及 wpa_supplicant_scan 调用。
那么,最先调用的地方肯定是 wpa_supplicant_driver_init。

wpa_supplicant_scan
除了在 wpa_supplicant 调用外(即wpa_supplicant_req_scan 注册到eloop, 实际被
eloop调用的),还赋给struct wpa_sm_ctx 的 scan 指针给state machine 调用。

wpa_supplicant_get_scan_results  还被 
wpa_supplicant_ctrl_iface_scan_results , wpa_supplicant_get_beacon_ie调用。

wpa_supplicant_ctrl_iface_init 在eloop 注册了 wpa_supplicant_ctrl_iface_receive,

wpa_supplicant_ctrl_iface_receive 调用 wpa_supplicant_ctrl_iface_process ,
process 再调用 wpa_supplicant_ctrl_iface_scan_results

\section{关联热点}
主要是 wpa_supplicant_associate , 是被 wpa_supplicant_event_scan_results 调用
的,wpa_supplicant_scan 在 Android 的配置下不会被执行到的。
        wpa_drv_set_auth_alg,
        wpa_drv_set_drop_unencrypted(wpa_s, use_crypt);
	wpa_supplicant_set_state(wpa_s, WPA_ASSOCIATING); 
	ret = wpa_drv_associate(wpa_s, \&params);



wpa_supplicant_set_suites

关联热点还有两个函数 wpa_driver_wext_event_wireless, wpa_supplicant_gen_assoc_event

wpa_driver_wext_init 向eloop 注册了套接字的回调,这个套接字就是s =
socket(PF_NETLINK, SOCK_RAW, NETLINK_ROUTE);,
而对应的回调函数就是wpa_driver_wext_event_receive。
此回调函数会调用 wpa_driver_wext_event_rtm_newlink, 后者调用
wpa_driver_wext_event_wireless 。

wpa_supplicant_scan 调用 wpa_supplicant_gen_assoc_event。
从关联角度我们重新再分析 wpa_supplicant_scan, 这个函数的行为跟配置文件很有关
系。在 Android ,ap_scan 设为 1 。所以,上面热点扫描就没有讨论这两种情况,这里
也不作过多讨论。

\section{验证和加密}
wpa_supplicant_req_auth_timeout  被调用的地方依次是\footnote{从日志上看,三个地方
都执行了wpa_supplicant_req_auth_timeout}:
wpa_supplicant_associate,
wpa_supplicant_event_assoc, 
wpa_supplicant_rx_eapol.
wpa_supplicant_associate 别 wpa_supplicant_event_scan_results 调用,
wpa_supplicant_rx_eapol 是在 l2_packet_receive 调用的, 因为 rx_callback 在
l2_packet 初始化时被赋值为wpa_supplicant_rx_eapol 。
l2_packet 出事化函数是 l2_packet_init, 她创建的套接字是用来获取原始的 L2 数据,见
\autoref{sockraw} ,
当然, 她关心的是 (即第三个参数protocol 的值)ETH_P_EAPOL (0x888e) 和 ETH_P_RSN_PREAUTH (0x88c7) 类型的数据包
   前者在wpa_supplicant.c, 后者在 preauth.c 


\section{关闭}

\begin{lstlisting}
wpa_driver_wext_deinit

	wpa_driver_wext_set_bssid(drv, (u8 *) "\x00\x00\x00\x00\x00\x00");
\end{lstlisting}


\section{调试}
'adb pull /proc/config.gz .'
'gunzip config.gz'

android wifi常用调试命令:
1、ifconfig wlan0 up
2、iwlist wlan0 scan
3、iwconfig wlan0 essid karl key 0123456789

common.h

\begin{lstlisting}
void android_printf(int level, char *format, ...);

//when debug, replace MSG_INFO with MSG_DEBUG.
#define wpa_printf(level, ...) \
        do {                                            \
            if ((level) >= MSG_INFO) {                  \
                android_printf((level), __VA_ARGS__);   \
            }                                           \
        } while (0)

#else /* ANDROID */
\end{lstlisting}


Android.mk 首先包括了:
\begin{lstlisting}
include $(LOCAL_PATH)/.config
\end{lstlisting}

在.config 配置了CONFIG_DRIVER_WEXT,
但对CONFIG_L2_PACKET未作配置,那么使用CONFIG_L2_PACKET 默认值,其默认值是
IEEE8021X_EAPOL 配置
CONFIG_NO_WPA2 未配置



\begin{lstlisting}
ifndef CONFIG_L2_PACKET
CONFIG_L2_PACKET=linux
endif

OBJS += l2_packet_$(CONFIG_L2_PACKET).c

\end{lstlisting}
L2 packet 是 l2_packet_linux.c

wpa cli 的主程序是文件wpa_cli.c 。


除了 wpa_cli 程序手工控制wpa_supplicant/wifi 

android 中 wifi 的HAL 会调用 wpa_ctrl 使用的是 libnetutils
libwpa_client 库, wpa_supplicant 编译了这个共享库。

main.c 是 wpa_supplicant 主程序。
wpa_supplicant.c 是主要初始化的。
wpa.c 是 状态机, sm 即 State Machine。

mlme.c 未编译,就是不使用。所以, wpa_supplicant 里的use_client_mlme 是为0.


\section{动态主机配置协议服务}

system/core/libnetutils/dhcp_utils.c

external/dhcpcd


frameworks/base/wifi/

frameworks/base/core/java/android/net

\section{Wifi 基本知识}

Wireless Card: Should be self explanitory... handles sending/receiving wireless packets

Linux Kernel: The linux kernel contains the driver for the wireless card, the
mac80211 subsystem which handles packet generation and scheduling, and the
nl80211 subsystem, which handles configuring the wireless interfaces for
userspace

libnl: libnl is the transport layer used to communicate with the kernel via netlink

udev: udev is the facility the kernel uses to pass events/calls to crda

iw: iw is a userspace utility that we can use to test that libnl is working
correctly, as well as create additional virtual wireless interfaces on the
wireless card

crda: crda is a userspace program that the kernel queries (through udev) to
find what channels/frequencies are usable, and at what powers. This moves the
information out of static tables maintained in kernel to userspace, which
allows them to be updated without reloading drivers/rebooting

Wireless Regulatory Database: database of allowable frequencies and transmit
power levels used by crda

hostapd: This is the daemon that handles generation of beacons and other
wireless packets, as wel as wpa-psk, wpa2, etc encryptions.

无线网卡:操作接收/发达无线包

linux核:linux核包涵了无线网卡的驱动,mac80211子系统 处理包的产生与时序安排。nl80211处理在用户空间配置无线接口  cfg80211 

libnl:通过netlink与内核对话的传输层

udev:udev是内核通过events/calls到 crda的设施

iw:是我们用来测试 libnl是否正确工作,以及在无线网卡上创建额外的虚拟无线接口的用户空间工具 。

crda:是内核通过udev来查找什么频段/频率是有效的,在什么强度的用户空间程序。这个从内
核维护的静态表中移到用户空间,它可以不需要重载驱动和重启的情况下更新。

无线控制数据库:crda使用允许的频率和传输强度等级的数据库

hostapd:这是个用来操作信号的产生和其它无线包,以及wpa-psk, wpa2等加密的守护进程

softMAC是一个用于描述无线网卡的类型的一个术语。


\begin{tabular}{|p{0.3\textwidth}|p{0.6\textwidth}|}\hline
Ad-hoc & 不需要 AP,各天线设备p2p(点对点)方式直接进行通讯。\\\hline
Infra(Infrastructure) &通过接入点(AP) 实现各无线设备之间通信的方式。 \\\hline
\end{tabular}


\section{网络性能测量}
Android 集成了工具 netperf,此工具包有两个应用:netperf 和 netserver,代码放在
: external/netperf。

Netperf是一种网络性能的测量工具,主要针对基于TCP或UDP的传输。Netperf根据应用的
不同,可以进行不同模式的网络性能测试,即批量数据传输(bulk data transfer)模式
和请求/应答(request/reponse)模式。Netperf测试结果所反映的是一个系统能够以多快
的速度向另外一个系统发送数据,以及另外一个系统能够以多块的速度接收数据。

Netperf工具以client/server方式工作。server端是netserver,用来侦听来自client端的
连接,client端是netperf,用来向server发起网络测试。在client与server之间,首先
建立一个控制连接,传递有关测试配置的信息,以及测试的结果;在控制连接建立并传递
了测试配置信息以后,client与server之间会再建立一个测试连接,用来来回传递着特殊
的流量模式,以测试网络的性能。


\newpage

\begin{appendices}
    \section{linux sock_raw原始套接字编程}\label{sockraw}
sock_raw原始套接字编程可以接收到本机网卡上的数据帧或者数据包,对与监听网络的流
量和分析是很有作用的.一共可以有3种方式创建这种socket
\begin{enumerate*}
    \item socket(AF_INET, SOCK_RAW, IPPROTO_TCP|IPPROTO_UDP|IPPROTO_ICMP)
        
        发送接收ip数据包

    \item socket(PF_PACKET, SOCK_RAW, htons(ETH_P_ALL))
        
        发送接收以太网数据帧, 注意第三个参数可以取ETH_P_IP, ETH_P_ARP,
        ETH_P_ALL, 但是不能用或运算(|)连接

    \item socket(AF_INET, SOCK_PACKET, htons(ETH_P_IP|ETH_P_ARP|ETH_P_ALL))
        
        过时了,不要用啊
\end{enumerate*}

理解一下SOCK_RAW的原理, 比如网卡收到了一个 14+20+8+100+4 的udp的以太网数据帧.

首先,网卡对该数据帧进行硬过滤(根据网卡的模式不同会有不同的动作,如果设置了
promisc混杂模式的话,则不做任何过滤直接交给下一层输入例程,否则非本机mac或者广播
mac会被直接丢弃).按照上面的例子,如果成功的话,会进入ip输入例程.但是在进入ip输入
例程之前,系统会检查系统中是否有通过socket(AF_PACKET, SOCK_RAW, ..)创建的套接字
.如果有的话并且协议相符,在这个例子中就是需要ETH_P_IP或者ETH_P_ALL类型.系统就给
每个这样的socket接收缓冲区发送一个数据帧拷贝.然后进入下一步.

其次,进入了ip输入例程(ip层会对该数据包进行软过滤,就是检查校验或者丢弃非本机ip
或者广播ip的数据包等,具体要参考源代码),例子中就是如果成功的话会进入udp输入例程
.但是在交给udp输入例程之前,系统会检查系统中是否有通过socket(AF_INET, SOCK_RAW,
..)创建的套接字.如果有的话并且协议相符,在这个例子中就是需要IPPROTO_UDP类型
.系统就给每个这样的socket接收缓冲区发送一个数据帧拷贝.然后进入下一步.

最后,进入udp输入例程 ...

如果校验和出错的话,内核会直接丢弃该数据包的.而不会拷贝给sock_raw的套
接字,因为校验和都出错了,数据肯定有问题的包括所有信息都没有意义了.

进一步分析他们的能力.
1. socket(AF_INET, SOCK_RAW, IPPROTO_UDP);

该套接字可以接收协议类型为(tcp udp icmp等)发往本机的ip数据包,从上面看的就是20+8+100.

\begin{itemize*}
    \item 不能收到非发往本地ip的数据包(ip软过滤会丢弃这些不是发往本机ip的数据包).
    \item 不能收到从本机发送出去的数据包.
    \item 发送的话需要自己组织tcp udp icmp等头部.可以setsockopt来自己包装ip头部
\end{itemize*}
这种套接字用来写个ping程序比较适合
     
2. socket(PF_PACKET, SOCK_RAW, htons(x));

这个套接字比较强大,创建这种套接字可以监听网卡上的所有数据帧.从上面看就是
20+20+8+100.最后一个以太网crc从来都不算进来的,因为内核已经判断过了,对程序来说
没有任何意义了.

\begin{itemize*}
    \item 能接收发往本地mac的数据帧
    \item 能接收从本机发送出去的数据帧(第3个参数需要设置为ETH_P_ALL)
    \item 能接收非发往本地mac的数据帧(网卡需要设置为promisc混杂模式)
\end{itemize*}

协议类型一共有四个 \marginpar{ linux/if_ether.h}

\begin{tabular}{|p{0.15\textwidth}|p{0.1\textwidth}|p{0.6\textwidth}|}\hline
ETH_P_IP  &  0x0800  & 只接收发往本机mac的ip类型的数据帧 \\\hline
ETH_P_ARP &  0x0806  & 只接受发往本机mac的arp类型的数据帧\\\hline
ETH_P_ARP &  0x8035 & 只接受发往本机mac的rarp类型的数据帧\\\hline
ETH_P_ALL &  0x3    &  接收发往本机mac的所有类型ip arp rarp的数据帧, 接收从本机
发出的所有类型的数据帧.(混杂模式打开的情况下,会接收到非发往本地mac的数据帧
)\\\hline
\end{tabular}

发送的时候需要自己组织整个以太网数据帧.所有相关的地址使用struct sockaddr_ll 而
不是struct sockaddr_in(因为协议簇是PF_PACKET不是AF_INET了),比如发送给某个机器,
对方的地址需要使用struct sockaddr_ll。这种socket大小通吃,强悍。

下面是一段相关的代码:
\begin{lstlisting}[language=C,]
#include <sys/types.h>
#include <sys/ioctl.h>
#include <net/if.h>
#include <netpacket/packet.h>
#include <sys/socket.h>
#include <netinet/in.h>
#include <arpa/inet.h>

.. .. ..
int sockfd = socket(PF_PACKET, SOCK_RAW, htons(ETH_P_ALL));
struct sockaddr_ll sll;
memset( &sll, 0, sizeof(sll) );
sll.sll_family = AF_PACKET;
struct ifreq ifstruct;
strcpy(ifstruct.ifr_name, "eth0");
ioctl(sockfd, SIOCGIFINDEX, &ifstruct);
sll.sll_ifindex = ifstruct.ifr_ifindex;
sll.sll_protocol = htons(ETH_P_ALL);
if(bind(fd, (struct sockaddr *) &sll, sizeof(sll)) == -1 ) {
       perror("bind()");
    ...
int set_promisc(char *interface, int fd) {
        struct ifreq ifr;
        strcpy(ifr.ifr_name, interface);
        if(ioctl(fd, SIOCGIFFLAGS, &ifr) == -1) {
                 perror("iotcl()");
                 return -1;
        }
        ifr.ifr_flags |= IFF_PROMISC;
        if(ioctl(fd, SIOCSIFFLAGS, &ifr) == -1) {
                 perror("iotcl()");
                 return -1;
         }
         return 0;
}

int unset_promisc(char *interface, int fd) {
        struct ifreq ifr;
        strcpy(ifr.ifr_name, interface);
        if(ioctl(fd, SIOCGIFFLAGS, &ifr) == -1) {
                 perror("iotcl()");
                 return -1;
        }
        ifr.ifr_flags &= ~IFF_PROMISC;
        if(ioctl(fd, SIOCSIFFLAGS, &ifr) == -1) {
                 perror("iotcl()");
                 return -1;
         }
         return 0;
}
\end{lstlisting}
3. socket(AF_INET, SOCK_PACKET, htons(ETH_P_ALL))这个最好不要用,反正我不用...
总结使用方法:
1.只想收到发往本机某种协议的ip数据包的话用第一种就足够了
2. 更多的详细的内容请使用第二种.包括ETH_P_ALL参数和混杂模式都可以使它的能力不断的加强.
ps:很多自己的想法.虚拟机测试环境.有错欢迎指出交流


\section{Netlink 协议}
\begin{tabular}{| p{0.12\textwidth} | p{0.28\textwidth} |p{0.5\textwidth} |}\hline
   文件系统 & 路径 & 作用\\\hline
procfs & /proc  & 主要用于揭示进程相关的信息.\\ \hline

sysfs & /sys & 用于揭示 devices, drivers, buses 等信息; 严格遵循
one-value-per-file 规则.\\\hline

debugfs & /sys/kernel/debug/ & 没有任何规则要求, 用户可以放入任何信息; 主
要用于调试目的. \\\hline

\end{tabular}

Netlink 相对于系统调用,ioctl 以及 /proc 文件系统而言具有以下优点:

\begin{enumerate*} 
    \item 为了使用 netlink,用户仅需要在 include/linux/netlink.h 中增加一个新
        类型的 netlink 协议定义即可, 如 \#define NETLINK_MYTEST 17 然后,内核
        和用户态应用就可以立即通过 socket API 使用该 netlink 协议类型进行数据
        交换。但系统调用需要增加新的系统调用,ioctl 则需要增加设备或文件, 那
        需要不少代码,proc 文件系统则需要在 /proc 下添加新的文件或目录,那将使
        本来就混乱的 /proc 更加混乱。
    \item  netlink是一种异步通信机制,在内核与用户态应用之间传递的消息
        保存在socket缓存队列中,发送消息只是把消息保存在接收者的socket的接收队列,
        而不需要等待接收者收到消息,但系统调用与 ioctl 则是同步通信机制,如果传递
        的数据太长,将影响调度粒度。
    \item 使用 netlink 的内核部分可以采用模块的方式实现,使用 netlink 的应用部分
        和内核部分没有编译时依赖,但系统调用就有依赖,而且新的系统调用的实现必须
        静态地连接到内核中,它无法在模块中实现,使用新系统调用的应用在编译时需要
        依赖内核。

    \item netlink 支持多播,内核模块或应用可以把消息多播给一个netlink组,属于该
        netlink 组的任何内核模块或应用都能接收到该消息,内核事件向用户态的通知机
        制就使用了这一特性,任何对内核事件感兴趣的应用都能收到该子系统发送的内核事件。

    \item 内核可以使用 netlink 首先发起会话,但系统调用和 ioctl 只能由用户应用发起调用。

    \item netlink 使用标准的 socket API,因此很容易使用,但系统调用和 ioctl则需要专门的培训才能使用。
\end{enumerate*}

DebugFS,顾名思义,是一种用于内核调试的虚拟文件系统,内核开发者通过debugfs和用
户空间交换数据。类似的虚拟文件系统还有procfs和sysfs等,这几种虚拟文件系统都并
不实际存储在硬盘上,而是Linux内核运行起来后才建立起来。

通常情况下,最常用的内核调试手段是printk。但printk并不是所有情况都好用,比如打
印的数据可能过多,我们真正关心的数据在大量的输出里不是那么一目了然;或者我们在
调试时可能需要修改某些内核变量,这种情况下printk就无能为力,而如果为了修改某个
值重新编译内核或者驱动又过于低效,此时就需要一个临时的文件系统可以把我们需要关
心的数据映射到用户空间。在过去,procfs可以实现这个目的,到了2.6
时代,新引入的sysfs也同样可以实现,但不论是procfs或是sysfs,用它们来实现某些
debug的需求,似乎偏离了它们创建的本意。比如procfs,其目的是反映进程的状态信息
;而sysfs主要用于Linux设备模型。不论是procfs或是sysfs的接口应该保持相对稳定,
因为用户态程序很可能会依赖它们。当然,如果我们只是临时借用procfs或者sysfs来作
debug之用,在代码发布之前将相关调试代码删除也无不可。但如果相关的调试借口要在
相当长的一段时间内存在于内核之中,就不太适合放在procfs和sysfs里了。故此,
debugfs应运而生。

中断上下文, 进程上下文

proc文件系统关注的是进程信息,/sysfs关注是one-value-per-file策略集,而Debugfs
文件系统没有如此多限制,可是任何内核要输出的信息。

mount -t debugfs none /sys/kernel/debug


sysctl:
he sysctl infrastructure is designed to configure kernel parameters at run
time. The sysctl interface is heavily used by the Linux networking subsystem.
It can be used to configure some core kernel parameters; represented as files
in /proc/sys/*. The values can be accessed by using cat(1), echo(1) or the
sysctl(8) commands. If a value is set by the echo command it only persists as
long as the kernel is running, but gets lost as soon as the machine is
rebooted. In order to change the values permanently they have to be written to
the file /etc/sysctl.conf. Upon restarting the machine all values specified in
this file are written to the corresponding files in /proc/sys/.

\begin{lstlisting}[language=C, caption= NetLink 获取网卡信息]
#include <stdio.h>
#include <string.h>
#include <errno.h>
#include <sys/socket.h>
#include <sys/types.h>
#include <unistd.h>
#include <asm/types.h>
#include <linux/netlink.h>
#include <linux/rtnetlink.h>
#include <linux/netdevice.h>
#include <net/if_arp.h>
#include <netinet/ether.h>
#include <netinet/if_ether.h>

#if 1
char * ether_ntoa_r (const struct ether_addr *addr, char *buf)
{
  sprintf (buf, "%x:%x:%x:%x:%x:%x",
       addr->ether_addr_octet[0], addr->ether_addr_octet[1],
       addr->ether_addr_octet[2], addr->ether_addr_octet[3],
       addr->ether_addr_octet[4], addr->ether_addr_octet[5]);
  return buf;
}

char * ether_ntoa (const struct ether_addr *addr)
{
  static char asc[18];

  return ether_ntoa_r (addr, asc);
}
#endif

int main()
{
    int nSocket, nAttrLen;
    unsigned nLen;
    char szBuffer[4096];
    struct {
        struct nlmsghdr nh;
        struct ifinfomsg ifi;
    }struReq;
    struct sockaddr_nl struAddr;
    struct nlmsghdr *pstruNL;
    struct ifinfomsg *pstruIF;
    struct rtattr *pstruAttr;
    struct net_device_stats *pstruInfo;
    struct ether_addr *pstruEther;

    /*
     * 创建一个PF_NETLINK的SOCKET,使用NETLINK_ROUTE协议
     */
    nSocket = socket(PF_NETLINK, SOCK_DGRAM, NETLINK_ROUTE);
    if(nSocket < 0)
    {
        fprintf(stderr, "创建SOCKET错误:%s\n", strerror(errno));
        return -1;
    }

    /*
     * 绑定地址
     */
    memset(&struAddr, 0, sizeof(struAddr));
    struAddr.nl_family = AF_NETLINK;
    struAddr.nl_pid = getpid();
    struAddr.nl_groups = 0;
    if(bind(nSocket, (struct sockaddr *)&struAddr, sizeof(struAddr)) < 0)
    {
        fprintf(stderr, "绑定SOCKET错误:%s\n", strerror(errno));
        return -1;
    }

    /*
     * 发送一个请求
     */
    memset(&struReq, 0, sizeof(struReq));
    struReq.nh.nlmsg_len = NLMSG_LENGTH(sizeof(struReq));
    struReq.nh.nlmsg_type = RTM_GETLINK;
    struReq.nh.nlmsg_flags = NLM_F_REQUEST | NLM_F_DUMP;
    struReq.ifi.ifi_family = AF_UNSPEC;
    memset(&struAddr, 0, sizeof(struAddr));
    struAddr.nl_family = AF_NETLINK;
    struAddr.nl_pid = 0;
    struAddr.nl_groups = 0;
    if(sendto(nSocket, &struReq, struReq.nh.nlmsg_len, 0,
                (struct sockaddr *)&struAddr, sizeof(struAddr)) < 0)
    {
        fprintf(stderr, "发送数据错误:%s\n", strerror(errno));
        return -1;
    }

    /*
     * 循环接收数据,直到超时
     */
    alarm(3);
    memset(szBuffer, 0, sizeof(szBuffer));
    while((nLen = recv(nSocket, szBuffer, sizeof(szBuffer), 0)))
    {
        alarm(0);
        pstruNL = (struct nlmsghdr *)szBuffer;
        /*
         * 判断是否继续有数据
         */
        while(NLMSG_OK(pstruNL, nLen))
        {
            /*
             * 数据已经获取完成
             */
            if(pstruNL -> nlmsg_type == NLMSG_DONE)
                break;
            if(pstruNL -> nlmsg_type == NLMSG_ERROR)
            {
                /*
                 * 发生一个错误
                 */
                struct nlmsgerr *pstruError;

                pstruError = (struct nlmsgerr *)NLMSG_DATA(pstruNL);
                fprintf(stderr, "发生错误[%s]\n",
                        strerror(-pstruError -> error));
                break;
            }

            /*
             * 下面通过宏获取数据
             */
            pstruIF = NLMSG_DATA(pstruNL);
            fprintf(stderr, "获取到设备[%d]信息\n", pstruIF -> ifi_index);
            fprintf(stderr, "\t设备类型:");
            switch(pstruIF -> ifi_type)
            {
                case ARPHRD_ETHER:
                    fprintf(stderr, "以太网\n");
                    break;
                case ARPHRD_PPP:
                    fprintf(stderr, "PPP拨号\n");
                    break;
                case ARPHRD_LOOPBACK:
                    fprintf(stderr, "环路设备\n");
                    break;
                default:
                    fprintf(stderr, "未知\n");
                    break;
            }
            fprintf(stderr, "\t设备状态:");
            if((pstruIF -> ifi_flags & IFF_UP )== IFF_UP)
                fprintf(stderr, " UP");
            if((pstruIF -> ifi_flags & IFF_BROADCAST) == IFF_BROADCAST)
                fprintf(stderr, " BROADCAST");
            if((pstruIF -> ifi_flags & IFF_DEBUG) == IFF_DEBUG)
                fprintf(stderr, " DEBUG");
            if((pstruIF -> ifi_flags & IFF_LOOPBACK) == IFF_LOOPBACK)
                fprintf(stderr, " LOOPBACK");
            if((pstruIF -> ifi_flags & IFF_POINTOPOINT) == IFF_POINTOPOINT)
                fprintf(stderr, " POINTOPOINT");
            if((pstruIF -> ifi_flags & IFF_RUNNING) == IFF_RUNNING)
                fprintf(stderr, " RUNNING");
            if((pstruIF -> ifi_flags & IFF_NOARP) == IFF_NOARP)
                fprintf(stderr, " NOARP");
            if((pstruIF -> ifi_flags & IFF_PROMISC) == IFF_PROMISC)
                fprintf(stderr, " PROMISC");
            if((pstruIF -> ifi_flags & IFF_NOTRAILERS) == IFF_NOTRAILERS)
                fprintf(stderr, " NOTRAILERS");
            if((pstruIF -> ifi_flags & IFF_ALLMULTI) == IFF_ALLMULTI)
                fprintf(stderr, " ALLMULTI");
            if((pstruIF -> ifi_flags & IFF_MASTER) == IFF_MASTER)
                fprintf(stderr, " MASTER");
            if((pstruIF -> ifi_flags & IFF_SLAVE) == IFF_SLAVE)
                fprintf(stderr, " SLAVE");
            if((pstruIF -> ifi_flags & IFF_MULTICAST) == IFF_MULTICAST)
                fprintf(stderr, " MULTICAST");
            if((pstruIF -> ifi_flags & IFF_PORTSEL) == IFF_PORTSEL)
                fprintf(stderr, " SLAVE");
            if((pstruIF -> ifi_flags & IFF_AUTOMEDIA) == IFF_AUTOMEDIA)
                fprintf(stderr, " AUTOMEDIA");
            if((pstruIF -> ifi_flags & IFF_DYNAMIC) == IFF_DYNAMIC)
                fprintf(stderr, " DYNAMIC");
            fprintf(stderr, "\n");

            /*
             * 下面通过宏获取属性
             */
            pstruAttr = IFLA_RTA(pstruIF);
            nAttrLen = NLMSG_PAYLOAD(pstruNL, sizeof(struct ifinfomsg));
            while(RTA_OK(pstruAttr, nAttrLen))
            {
                switch(pstruAttr->rta_type)
                {
                    case IFLA_IFNAME:
                        fprintf(stderr, "\t设备名称:%s\n",
                                (char *)RTA_DATA(pstruAttr));
                        break;
                    case IFLA_MTU:
                        fprintf(stderr, "\t设备MTU:%d\n",
                                *(unsigned int *)RTA_DATA(pstruAttr));
                        break;
                    case IFLA_QDISC:
                        fprintf(stderr, "\t设备队列:%s\n",
                                (char *)RTA_DATA(pstruAttr));
                        break;
                    case IFLA_ADDRESS:
                        if(pstruIF -> ifi_type == ARPHRD_ETHER)
                        {
                            pstruEther = (struct ether_addr *)
                                RTA_DATA(pstruAttr);
                            fprintf(stderr, "\tMAC地址:%s\n",
                                    ether_ntoa(pstruEther));
                        }
                        break;
                    case IFLA_BROADCAST:
                        if(pstruIF -> ifi_type == ARPHRD_ETHER)
                        {
                            pstruEther = (struct ether_addr *)
                                RTA_DATA(pstruAttr);
                            fprintf(stderr, "\t广播MAC地址:%s\n",
                                    ether_ntoa(pstruEther));
                        }
                        break;
                    case IFLA_STATS:
                        pstruInfo = (struct net_device_stats *)
                            RTA_DATA(pstruAttr);
                        fprintf(stderr, "\t接收信息:\n");
                        fprintf(stderr, "\t\t接收报文:%lu 字节:%lu\n",
                                pstruInfo -> rx_packets, pstruInfo -> rx_bytes);
                        fprintf(stderr, "\t\terrors:%lu dropped:%lu "
                                "multicast:%lu collisions:%lu\n",
                                pstruInfo -> rx_errors, pstruInfo -> rx_dropped,
                                pstruInfo -> multicast, pstruInfo -> collisions);
                        fprintf(stderr, "\t\tlength:%lu over:%lu crc:%lu "
                                "frame:%lu fifo:%lu missed:%lu\n",
                                pstruInfo -> rx_length_errors,
                                pstruInfo -> rx_over_errors,
                                pstruInfo -> rx_crc_errors,
                                pstruInfo -> rx_frame_errors,
                                pstruInfo -> rx_fifo_errors,
                                pstruInfo -> rx_missed_errors);
                        fprintf(stderr, "\t发送信息:\n");
                        fprintf(stderr, "\t\t发送报文:%lu 字节:%lu\n",
                                pstruInfo -> tx_packets, pstruInfo -> tx_bytes);
                        fprintf(stderr, "\t\terrors:%lu dropped:%lu\n",
                                pstruInfo -> tx_errors, pstruInfo -> tx_dropped);
                        fprintf(stderr, "\t\taborted:%lu carrier:%lu fifo:%lu"
                                " heartbeat:%lu window:%lu\n",
                                pstruInfo -> tx_aborted_errors,
                                pstruInfo -> tx_carrier_errors,
                                pstruInfo -> tx_fifo_errors,
                                pstruInfo -> tx_heartbeat_errors,
                                pstruInfo -> tx_window_errors);
                        break;
                    default:
                        break;

                }
                /*
                 * 继续下一个属性
                 */
                pstruAttr = RTA_NEXT(pstruAttr, nAttrLen);
            }
            /*
             * 继续下一个数据
             */
            pstruNL = NLMSG_NEXT(pstruNL, nLen);
        }
        memset(szBuffer, 0, sizeof(szBuffer));
        alarm(3);
    }
    return 0;
}
\end{lstlisting}
\end{appendices}


\begin{thebibliography}{源文件清单}
% 文档中新增交叉引用后,第一次执行 latex 或 pdflatex 编译命令时
% 会得到类似下面的警告信息。因为第一次编译只会扫描出有交叉引用的地
% 方,第二次编译才能得到正确结果。
\bibitem { } frameworks/base/core/java/android/net/NetworkStateTracker.java   
\bibitem { } frameworks/base/services/java/com/android/server/WifiService.java      
\bibitem { } frameworks/base/wifi/java/android/net/wifi/WifiStateTracker.java
\bibitem { } frameworks/base/wifi/java/android/net/wifi/WifiMonitor.java  
\bibitem { } frameworks/base/wifi/java/android/net/wifi/WifiNative.java
\bibitem { } \verb|frameworks/base/core/jni/android_net_wifi_Wifi.cpp| //WifiNative jni
\bibitem { } frameworks/base/core/java/android/net/NetworkUtils.java //DHCP   
\bibitem { } \verb|frameworks/base/core/jni/android_net_NetUtils.cpp| //NetworkUtils jni
\bibitem { } \verb|system/core/libnetutils/dhcpclient.c|
\bibitem { } \verb|system/core/libnetutils/dhcp_utils.c|
\bibitem { } \verb|hardware/libhardware_legacy/wifi/wifi.c|
\bibitem { } \verb|hardware/libhardware_legacy/include/libhardware_legacy/wifi.h|
\bibitem { } \verb|external/wpa_supplicant/main.c|
\bibitem { } \verb|external/wpa_supplicant/wpa.c|
\bibitem { } \verb|external/wpa_supplicant/os_unix.c|
\bibitem { } \verb|external/wpa_supplicant/drivers.c|
\bibitem { } \verb|external/wpa_supplicant/driver_wext.c|
\bibitem { } \verb|external/wpa_supplicant/wpa_supplicant.c|
\bibitem { } \verb|external/wpa_supplicant/ctrl_iface_unix.c|
\bibitem { } \verb|external/wpa_supplicant/ctrl_iface.c|
\bibitem { } \verb|external/wpa_supplicant/events.c|
\bibitem { } \verb|external/wpa_supplicant/eloop.c|
\bibitem { }  external/wpa_supplicant   rc4.c sha1.c md5.c md4.c 
    l2_package_linux.c preauth.c
    tls_openssl.c pmksa_cache.c eap_psk.c eap_psk_common.c eap_sake.c
    eap_sake_common.c eap_tls.c eap_tlv.c eap_ttls.c eap_pax.c
\end{thebibliography}

\end{document}%end fo document
