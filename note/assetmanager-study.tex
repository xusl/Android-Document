\documentclass[a4paper,11pt]{article}
%\usepackage{tocvsec2}
%\usepackage{titletoc}
%\usepackage{titlesec}
\usepackage{ifthen}
\usepackage{verbatim}
\usepackage{longtable, array, multicol,multirow}
\usepackage{enumitem, mdwlist}
\usepackage{lmodern}
\usepackage[T1]{fontenc}
\usepackage{textcomp}
\usepackage{underscore}%正文使用_
\usepackage{fancyhdr}
\pagestyle{fancy}
\usepackage[titletoc]{appendix}
\usepackage{makeidx}%索引
\usepackage[iso]{isodateo}%控制\today的格式
\usepackage{tikz}
\usetikzlibrary{arrows,decorations.pathmorphing,positioning,backgrounds,chains,fit,calc,%
petri,through,shapes}
\usepackage{graphics}
\usepackage{listings}
\usepackage{xcolor}
\lstset{language=C,
    breaklines=true,
    extendedchars=false,
    breakautoindent=true,
    breakindent=10pt,
    xleftmargin=30pt, 
    numbers=left, 
    basicstyle=\scriptsize,
    %numberstyle=\tiny,
    %keywordstyle=\tiny\color{blue!70}, 
    %commentstyle=\tiny\color{red!50!green!50!blue!50},
    showspaces=false,
    showstringspaces=false,
    escapeinside={(*@}{@*)} 
   %  frame=shadowbox,
  %  rulesepcolor=\color{red!20!green!20!blue!20}
  }
\lstdefinelanguage{KC}{language={C},morekeywords={uint32_t, size_t}}
\usepackage[CJKbookmarks=true, bookmarksnumbered=true,
colorlinks=true,% frenchlinks=true, raiselinks=true, xetex,, linktocpage=true 
citecolor=magenta, linkcolor=blue]{hyperref}
%\usepackage[CJKbookmarks=true, bookmarksnumbered=true,
%colorlinks=false, citecolor=magenta]{hyperref}
\usepackage[superscript]{cite}
\usepackage{xeCJK}
\setCJKmainfont{AR PL UKai CN}
\setCJKmonofont{文泉驿等宽正黑}
\punctstyle{banjiao}
%\setCJKfamilyfont{AR PL UKai CN}{kai}
\setmainfont{DejaVu Sans}
\setmonofont{AR PL UKai CN}
\title{Android 资源管理器分析}
\author{shecenon@gmail.com}
\date{\today}

\makeindex 

\begin{document}
\maketitle
\renewcommand{\refname}{参考文献}
\renewcommand\contentsname{目录}
\renewcommand\listfigurename{插图目录}
\renewcommand\listtablename{表格目录}
\renewcommand\indexname{索引}
\renewcommand\appendixname{附录}
\renewcommand\figurename{图}
\renewcommand\tablename{表}
\renewcommand{\lstlistlistingname}{代码列表集}
\renewcommand{\lstlistingname}{代码}

\def\equationautorefname{公式}%
\def\figureautorefname{图}%
\def\subfigureautorefname{子图}%
\def\sectionautorefname{节}%
\def\subsectionautorefname{子节}%
\def\subsubsectionautorefname{小节}%
\def\Itemautorefname{项}%
\def\tableautorefname{表格}%
\def\footnoteautorefname{脚注} 
\def\appendixautorefname{附录}

%\renewcommand\thefootnote{\thempfootnote}

\setdescription{labelsep=\textwidth}
%\setlength{\multicolsep}{3pt} 
%\multicolsep=\skip59
\setlength{\columnsep}{.8cm}
%\titlecontents{section}[1.5em]{}{\thecontentslabel}{}{\dotfill \contentspage}
%\titlecontents{section}[1.5em]{\color{blue}\normalsize\addvspace{1.0ex}}{\contentslabel{2em}\hspace∗{−0.0em}}{\hspace∗{−3.3em}}{\color{black}\titlerule ∗[0.8pc]{}\contentspage}
\setlength{\parskip}{1ex}
\headheight13.8pt
\tableofcontents
\makeatletter
\newcommand{\figcaption}{\def\@captype{figure}\caption}
\newcommand{\tabcaption}{\def\@captype{table}\caption}
\makeatother

\def\call#1{\\* {\indent #1 \\*\noindent}} 
%\newcommand{\call}[1]{\\* {\kern10pt #1 \\*\noindent}}
\newcommand{\various}[1]{{\color{cyan}\textit{#1}}}
%\newcommand{\binder}{{\color{red}binder }}
\def\binder{{\color{red}binder }}
\def\cref#1{$^{\scriptsize\ref{#1}}$}

\setlength{\leftmargin}{1.2em}     %左边界
\setlength{\parsep}{0ex}         %段落间距
\setlength{\topsep}{1ex}         %列表到上下文的垂直距离
\setlength{\itemsep}{0.5ex}        %条目间距
\setlength{\labelsep}{0.3em}     %标号和列表项之间的距离,默认0.5em
\setlength{\itemindent}{1.1em}    %标签缩进量
\setlength{\listparindent}{0em} %段落缩进量

Android 在应用程序的开发中大量使用了资源,他也把资源的使用发挥了极致。
资源管理的核心代码在frameworks/base/libs/utils 中,主要有 
Asset.cpp,AssetManager.cpp,AssetDir.cpp,ResourceTypes.cpp
以及基础StringPool.cpp, String16.cpp, String8.cpp。
这是资源的编解码库。
aapt 是生成资源的工具,也使用了这个库。而Android应用程序也是使用这个库来解析资
源。 这从utils目录下的Android.mk 就可以看出来,把编译了本地的库和目标机器的库
。

\section{资源管理器一瞥}
AssetManager 资源管理器
aapt 是比较复杂的,不适合入门。bootanimation
\footnote{frameworks/base/cmds/bootanimation} 有个简单的,可作示例:

\begin{lstlisting}
status_t BootAnimation::initTexture(Texture* texture,
                AssetManager& assets, const char* name) {
    Asset* asset = assets.open(name, Asset::ACCESS_BUFFER);
    if (!asset)
        return NO_INIT;
    SkBitmap bitmap;
    SkImageDecoder::DecodeMemory(asset->getBuffer(false),
            asset->getLength(), &bitmap, SkBitmap::kNo_Config,
            SkImageDecoder::kDecodePixels_Mode);
    asset->close();
    delete asset;
    . . .
}
\end{lstlisting}

这个name 就是路径,如 "images/android-logo-shine.png",因为这个不是Android
应用程序,这个资源就是"framework/framework-res.apk",即 AssetManager.cpp 的
kSystemAssets 值。

从这段代码看,就是使用 AssetManager 打开一个文件,得到的是 Asset,从Asset调用
getBuffer得到数据, 调用 getLength 得到数据长度,调用close 关闭资源,然后要删
除 delete 这个对象(因为动态分配的)。

\subsection{创建资源管理器}
BootAnimation 有个疑问就是 AssetManager 对象的不知如何创建的 ?
不过,android_util_AssetManager.cpp 的 android_content_AssetManager_init 给的
很详细:
\begin{lstlisting}
static void android_content_AssetManager_init(JNIEnv* env, jobject clazz)
{
    AssetManager* am = new AssetManager();
    if (am == NULL) {
        jniThrowException(env, "java/lang/OutOfMemoryError", "");
        return;
    }

    am->addDefaultAssets();

    LOGV("Created AssetManager %p for Java object %p\n", am, clazz);
    env->SetIntField(clazz, gAssetManagerOffsets.mObject, (jint)am);
}
\end{lstlisting}
AssetManager.java 的构造器都要调用 init 方法, ActivityThread.java, 
PackageParser.java 会创建 AssetManager 对象。

aapt 中 AssetManager 对象在什么地方创建? 就是 AssetManager 类型数据成员 
mIncludedAssets。%哦,对C++语言不熟了,声明了对象就创建了对象!

\subsection{加入资源包}
调用 addDefaultAssets 会把framework/framework-res.apk 加入资源管理器,
bootanimation 就调用了他。

addDefaultAssets 是对 addAssetPath 的包装,而应用程序自己的资源包就 使用
addAssetPath,对Android 应用来讲 PackageParser.java , ActivityThread.java
会使用到的。
对于PackageParser.java 来讲,他是要获取并解析 AndroidManifest.xml 中的内容。
对于ActivityThread.java 来讲,他把 AssetManager 加入到 Resources,应用会使用
Resources 获取资源的。 对于 Resources 来讲,framework-res.apk 在AssetManager
对象创建时就已经加入的资源管理器。

\subsection{获取资源文件}
AssetManager 提供了如下几个函数:\footnote{资源管理器中, Asset 代表的是一个文
件。 AssetDir 表示目录。}
\begin{enumerate*}
    \item open\\ 打开apk中未压缩的文件。AssetManager 的open 函数会在name 前加上 "assets"。比如,
        bootanimation 中打开文件"images/android-logo-shine.png" 就是
        获取 framework-res.apk 中的压缩文件assets/images/android-logo-shine.png
    \item openNonAsset \\ 获取apk经压缩处理的文件。比如经过 android_util_AssetManager.cpp,
        AssetManager.java 的封装,成为 Android 层AssetManager 的
        openXmlResourceParser。 打开apk包里xml文件,并以解析器的形式返回,如
        PackageParser.java  的 parsePackage \footnote{JarFile 也可以获得apk包
        里的文件,见 collectCertificates}:
\begin{lstlisting}
XmlResourceParser parser = null;
AssetManager assmgr = null;
boolean assetError = true;
try {
    assmgr = new AssetManager();
    int cookie = assmgr.addAssetPath(mArchiveSourcePath);
    if(cookie != 0) {
        parser = assmgr.openXmlResourceParser(cookie, "AndroidManifest.xml");
        assetError = false;
    } else {
        Log.w(TAG, "Failed adding asset path:"+mArchiveSourcePath);
    }
}
\end{lstlisting}

    \item getResources \\ 对 getResTable 的封装,获取 apk 包中的
        resources.arsc 文件的资源接口。 AssetManager 的getResTable 是没有被
        外面调用的,因为申明为private的成员函数了(AssetManager.h)。

    \item openNonAssetDir \\ aapt 工具使用。
\end{enumerate*}

\subsection{获取资源的内容}
这些工作在ResourceTypes.cpp, Asset.cpp 中完成的,对不同类型的资源,使用不同的方法。

\begin{tabular}{lll}
    AssetManager 接口& 处理资源接口 & 接口文件 \\\hline
    open & Asset & Asset.c \\\hline
   \multirow{2}{*}{openNonAsset}  & ResXMLTree & ResourceTypes.cpp \\
    &Asset &Asset.c\\\hline
    getResources & ResTable & ResourceTypes.cpp \\\hline
     openDir& AssetDir &AssetDir.cpp \\\hline
\end{tabular}
\\Android 编译后,apk 中文件,有AssetManager 的可以分为以下类别:
\begin{enumerate*}
    \item classes.dex \\由虚拟机解析。
    \item resources.arsc \\由 ResTable 解析,ResTable_header、ResTable_package
        、ResStringPool_header 等数据结构。
    \item XML文件\\包括 AndroidManifest.xml 以及res/目
        录下layout、xml、anim、drawable*、color等目录里xml。
        \\XML文件在apk中以ResXMLTree形式组织,以方便在设备中解析(ResXMLParser)
        。涉及 ResXMLTree_header、ResXMLTree_node、
        ResXMLTree_attribute 等数据结构。
    \item 原始文件,包括asset目录, res/目录下 drawable 等目录下的png文件、raw
        目录。 
    \item META-INF 目录下文件。
\end{enumerate*}
上述 resources.arsc、XML文件 的数据是按数据块(chunk)划分的树形结构,
每个数据块都以 struct ResChunk_header 开头,占8个字节:
\begin{description*}
    \item [type] (uint16_t) 表示数据块类型,
    \item [headerSize] (uint16_t)表示数据块头的大小,
        他不是struct ResChunk_header 本身大小,而是包含
        struct ResChunk_header 的数据结构的大小;
    \item [size] (uint32_t)表示数据块的大小,包括ResChunk_header
        所在数据结构本身大小,其后附件的一些索引数组,数据以及
        其包含的其他数据块(chunk)的大小。所以,组织结构相当于一
        个树形结构,一个跟节点,中间节点可能只是数据,也可能包含
        其他的数据块。
\end{description*}

\begin{minipage}{\linewidth}
\tabcaption{重要数据块的一些特征}
\begin{tabular}{|llc|}\hline
    数据结构 & 头部长度 & .arsc起始数据\\
             & (headerSize) & (type+headerSize)\\\hline
  ResStringPool_header & 0x1c(28) & 0100 1C00 \\\hline
  ResTable_header & 0x0c(12) & 0200 0C00 \\
  ResTable_package & 0x11c(284) & 0002 1C01 \\
  ResTable_type \footnote{ResTable_config 含有的成员函数不占据内存的, 所以只有
  32字节。}
  & 0x34(52) & 0102 3400\\
  ResTable_typeSpec & 0x10(16) & 0202 1000\\ \hline
  ResXMLTree_header & 0x08(8)& 0300 0800 \\ 
  ResXMLTree_node & 0x10(16)& 0[0-4]01 1000\\
  \hline
\end{tabular}
\end{minipage}

数据结构名以 ResTable 开头的适用于 resources.arsc 文件,这个文件以0200 0C00 (
ResTable_header)开始的。

数据结构名以 ResXMLTree 开头的适用于 xml 文件,这种xml文件以 0300 0800
(ResXMLTree_header)开始的。

\section{资源表数据结构}
AssetManager::getResTable 创建的大概流程:
\begin{lstlisting}
ResTable* rt = mResources;
Asset* ass = NULL;
ass = const_cast<AssetManager*>(this)->
         openNonAssetInPathLocked("resources.arsc",
         Asset::ACCESS_BUFFER,
         ap);
mResources = rt = new ResTable();
rt->setParameters(mConfig); //in updateResourceParamsLocked();
rt->add(ass, (void*)(i+1), !shared);
\end{lstlisting}
ResTable 解析 resources.arsc 的成员函数是 add。

\subsection{ResTable_header}
从数据块角度,resources.arsc是一个 ResTable_header,里面包了一个 ResStringPool_header 数据块和
一个或多个ResTable_package 数据块,其中 ResTable_package 的个数包含在 ResTable_header
数据成员 packageCount。见ResourceTypes.h 中数据结构 ResTable_header 定义的说明。
ResTable_package的类型是RES_TABLE_TYPE(即其包含的ResChunk_header 的 type 值)。

\subsection{ResStringPool_header}
ResStringPool_header 数据块解析:
\begin{itemize*}
    \item stringStart 字符串池数据起始地址的相对于 ResStringPool_header 开始地址的偏移量
    \item stringCount 字符串池的字符串数量。
\end{itemize*}

ResStringPool_header 结尾与字符串池数据存放地址之间,有个数组,存放每个字符
串起始地址相对字符串池数据起始地址的偏移量。

字符串以utf-16编码存放(对ascii字符讲就是占2个字节),各个
字符串连接在一起。每个字符串以其字符长度(也是uint16_t数
据存放)开始,以0x0000结束,长度不包含结尾的0x0000。但是也可
以在flags 设置UTF8_FLAG (1<<8),这样字符串就是以UTF-8编码。

 \subsection{ResTable_package}
ResTable_package 旗下包含了2个ResStringPool_header,
\begin{description*}
    \item [资源类型\label{restype}符号表] 例如 attr, drawable, string, layout 等各种资源类型。
    \item [资源关键字符号表] 比如,values/strings.xml有下面一条: 
        \\\hbox{<string name="hello">Hello World, Activity!</string>}\\
        其中, "hello" 就放在关键字符号表,而 "Hello World, Activity!" 放在
        ResTable_header 中的字符串池中。 对于 drawable, layout中的xml 文件和图片,其文
        件名放在关键字符号表(不包含扩展名),文件的路径放在ResTable_header 中的字符串池
        中,例如,layout/main.xml,关键字符号表包含字符串"main" ,ResTable_header 
        中的字符串池包含字符串"layout/main.xml"。
\end{description*}
ResTable_package 中这两个字符串池后面跟一个或多个ResTable_type 
和 ResTable_typeSpec 数据结构。
结构是以一个 ResTable_typeSpec + 掩码数组 + ResTable_type 组成一个单元,这样一
个单元可以描述一个R.java的内部类,比如 R.drawable,R.string等。 R.java 里有几个内部类,就
有几个这样的单元。R.java中,attr类总是存在的,不管其有无属性,其他的有属性则存在,无
则不存在。不过,attr没有属性时,仅有ResTable_typeSpec,没有掩码数组 + ResTable_type。

\subsubsection{ResTable_typeSpec}
每个资源类型\ref{restype}(resource type)必须有一个ResTable_typeSpec.
其id 从1 开始, 1,2,3,4,依次递增。 entryCount 可以为0,这样的 typeSpec 也
会出现的,当为0时,后面没有数组,若不为0,typeSpec后面就跟着entryCount个
uint32_t数据,一个unit32_t 表示一个配置的掩码,表示有是否有多个配置以及是否是
公开的。

typeSpec 表示的是R.java 中R的内部类, entryCount 表示这些内部类的属性的个数,
掩码就是表明这些属性在有几个不同配置(或者说是值)。
例如:下面的R.java 对应的 typeSpec
%\begin{multicols}{2}
\begin{lstlisting}[basicstyle=\tiny]
public final class R {
    public static final class attr {
    }
    public static final class drawable {
        public static final int icon=0x7f020000;
        public static final int img=0x7f020001;
        public static final int star=0x7f020002;
    }
    public static final class layout {
        public static final int main=0x7f030000;
    }
    public static final class string {
        public static final int app_name=0x7f040001;
        public static final int hello=0x7f040000;
    }
}
\end{lstlisting}
%\vfill
%\columnbreak
\begin{tabular}{lll}
    内部类 & id & entryCount \\
    attr & 01 & 0 \\
    drawable & 02 & 03 \\
    layout & 03 & 01 \\
    string &04 & 02\\
\end{tabular}
%\end{multicols}
还可以注意到, id值与内部类属性值的关联, 比如 drawable 的 id 为02,那drawable
属性值的4-5位便是02, 另外,id - 1 就是 ResTable_package资源类型符号表的索引,
通过他可以确定是att, layout, string等值.
. entryCount 就是每个内部类的属性个数。

\subsubsection{ResTable_type}
每个 ResTable_typeSpec 后面跟的就是 ResTable_type,这两个数据结构都有 id 和
entryCount数据成员,对这样一对的数据来讲,其id值相等的,entryCount的值也是一样
的。
ResTable_type后面跟着 4 * entryCount个字节的数据,是后面 ResTable_entry 数组
元素的偏移量,其值是: 数组下标 * sizeof(ResTable_entry)。
再后面就是entryCount 个 ResTable_entry数据,这些ResTable_entry 对起始地址与
ResTable_type 的起始地址的偏移值就是 entriesStart。

$entriesStart = sizeof(ResTable\_type) + sizeof(uint32\_t) * entryCount;$

ResourceTable::flatten\cref{restblflatten}
\begin{lstlisting}
const size_t typeSize = sizeof(ResTable_type) + sizeof(uint32_t)*N;
.. .. ..

const size_t typeStart = data->getSize();

ResTable_type* tHeader = (ResTable_type*)
    (((uint8_t*)data->editData(typeStart+typeSize)) + typeStart);
if (tHeader == NULL) {
    .. .. ..
    return NO_MEMORY;
}

memset(tHeader, 0, sizeof(*tHeader));
tHeader->header.type = htods(RES_TABLE_TYPE_TYPE);
tHeader->header.headerSize = htods(sizeof(*tHeader));
tHeader->id = ti+1;
tHeader->entryCount = htodl(N);
tHeader->entriesStart = htodl(typeSize);
tHeader->config = config;
\end{lstlisting}

ResTable_type 包含一个 ResTable_config 类型成员 config, 对同一个资源
类型来讲,可能有多个 ResTable_type 的数据块,每个代表不同的配置。

\subsubsection{ ResTable_entry}
8个字节长度,开始数据是0x0800,紧接着可能的取值是 0x0000 到 0x0003。
key 虽是ResStringPool_ref,其实就是一个uint32_t类型的数值,作为
ResTable_header 中的字符串池的索引值,从0开始计数的。 从R.java来讲,
ResTable_entry 表示的是其内部类的各个属性, 对于 drawable, 通过key我们可以定
位这个图片的位置,如 res/drawable/icon.png。
一个ResTable_entry 跟一个Res_value 或 ResTable_map。

Res_value 
也是8个字节长度, 开始数据也是 0x0800。
如果 Res_value 开始的值为0800 0000,ResTable_entry 开始的值为 0800 0003,那么
Res_value 数据成员data的值,和 ResTable_entry数据成员key的值相等的。

\subsubsection{ResTable_map_entry}
ResTable_map_entry 是 ResTable_map的派生类,
ResTable_map_entry 对应的值是 ResTable_map, 就像ResTable_entry 对应的值是
Res_value。

\section{资源表主要函数}
主要讲述的是 ResTable 类中的函数。
其实,生成资源表是 appt 负责,但是与 ResTable 也有关系的。这里不要混淆
ResourceTable 和ResTable 类了。
%ResTable 解析 resources.arsc 的成员函数是 add。
\begin{lstlisting}[language=xml,label=xmlexam,caption={Android XML 举例}]
<TextView
    android:text="@string/weekpicker_title"
    android:textAppearance="?android:attr/textAppearanceMedium"
    android:layout_gravity="center_horizontal"
    android:layout_width="wrap_content"
    android:layout_height="wrap_content" />

<TextView android:id="@+id/headerText"
    android:layout_width="match_parent"
    android:layout_height="0dip"
    android:layout_weight="1.0"
    android:gravity="center"
    android:textSize="18sp"/>

<View
     android:background="@*android:drawable/code_lock_top"
     android:layout_width="match_parent"
     android:layout_height="2dip" />
\end{lstlisting}
\subsection {stringToValue}\label{strtovalue}
ResTable::stringToValue 实际上只被 ResourceTable::stringToValue 调用,这是因为
她只用于构建资源表的,而不是解析/使用资源表的。 ResourceTable 只是aapt用于构造
ResTable 的类,与使用 apk 无直接关系。
此函数与Accessor\cref{accessorclass}也有比较紧密的联系。
在 ResourceTypes.h 中对其的注释是
\begin{verbatim}
// Convert a string to a resource value.  Handles standard "@res",
// "#color", "123", and "0x1bd" types; performs escaping of strings.
// The resulting value is placed in 'outValue'; if it is a string type,
// 'outString' receives the string.  If 'attrID' is supplied, the value is
// type checked against this attribute and it is used to perform enum
// evaluation.  If 'acccessor' is supplied, it will be used to attempt to
// resolve resources that do not exist in this ResTable.  If 'attrType' is
// supplied, the value will be type checked for this format if 'attrID'
// is not supplied or found.
\end{verbatim}

ResTable::stringToValue 对属性值作检测, 如属性值开头的
'@', '\#', '?', "true", "TRUE", "false", "FALSE"。
ResTable::collectString 对转义字符串以及上面提及的特殊字符会做处理。

\begin{tabular}{lll}
@  & 表示引用 & TYPE_REFERENCE\\
\# & 表示颜色 & TYPE_INT_COLOR_*\\
? & 表示属性 & TYPE_ATTRIBUTE\\
\end{tabular}

\subsubsection{有关'@'的说明}
\begin{tabular}{ll}
@null & 空值,没有引用值。\\
@+ & 如果资源不存在则创建。一般是android:id="@+id/myid"。\\
@* & 引用的公共的资源。\\
@  & 一般的引用,可以是私有的,也可以是公共。\\
\end{tabular}

\begin{lstlisting}[caption={stringToValue对@的处理代码},label=aterisk,]
bool createIfNotFound = false;
const char16_t* resourceRefName;
int resourceNameLen;
if (len > 2 && s[1] == '+') {
    createIfNotFound = true;
    resourceRefName = s + 2;
    resourceNameLen = len - 2;
} else if (len > 2 && s[1] == '*') {
    enforcePrivate = false;
    resourceRefName = s + 2;
    resourceNameLen = len - 2;
} else {
    createIfNotFound = false;
    resourceRefName = s + 1;
    resourceNameLen = len - 1;
}
\end{lstlisting}

通过变量createIfNotFound ,我们可以查到函数调用
\begin{lstlisting}
accessor->getCustomResourceWithCreation(package, type, name, createIfNotFound);
\end{lstlisting}
使用了此变量。所以,对于"@+",创建的新的资源项,就在getCustomResourceWithCreation中实现的。

搜索"'@'",注意带',有几个地方:
expandResourceRef, identifierForName, stringToValue, collectString有对其处理。

我们顺着\autoref{aterisk}往下看\label{derefer},
\begin{itemize*}
\item expandResourceRef 把节点属性值分解成包(package)、类型(type)、资源名(name),见
        \autoref{idforname}的分析,这个函数一定要返回成功,否则就退出了。
\item identifierForName 获得资源名对应的id,此函数最后会
    调用Res_MAKEID产生id,这个函数只
    是从已有的资源里获取资源的信息来提供id,并不会创建资
    源\footnote{创建资源请见ResourceTable::addEntry
    \cref{tableaddentry}},所以找不到资源时会返回id为0。

    当identifierForName 返回的id 不为0,accessor 参数为null时,stringToValue
    就返回这个id。 当 accessor 不为 null, 调用其
    getRemappedPackage,但是现在这个函数只是直接返回参数值,
    所以,最后其返回的值跟accessor为null时一样的。\autoref{getresid}的第二个
    实现也是有这种情形。

\item Accessor::getCustomResourceWithCreation 当identifierForName 返回id
为0,而accessor 不为null(一般这个是成立的,
因为Accessor是个抽象类,其具体派生类是ResourceTable)就调用这个函数
对应的就是ResourceTable::getCustomResourceWithCreation\cref{gcrwc}.

\end{itemize*}

\subsection{identifierForName}\label{idforname}
这里的name指的就是节点的属性值。
从函数结尾调用 Res_MAKEID 的参数来看,PackageGroup 相当于一个 Pakcage。

包名(packageEnd)以":"作为标志,类型名(typeEnd)以"/"作为标志。
下面的代码就是解析节点的值中的包名、类型以及入口,
\begin{lstlisting}[multicols=2,caption=identifierForName解析节点的代码]
// Figure out the package and type we are looking in...

const char16_t* packageEnd = NULL;
const char16_t* typeEnd = NULL;
const char16_t* const nameEnd = name+nameLen;
const char16_t* p = name;
while (p < nameEnd) {
    if (*p == ':') packageEnd = p;
    else if (*p == '/') typeEnd = p;
    p++;
}
if (*name == '@') name++;
if (name >= nameEnd) {
    return 0;
}

if (packageEnd) {
    package = name;
    packageLen = packageEnd-name;
    name = packageEnd+1;
} else if (!package) {
    return 0;
}

if (typeEnd) {
    type = name;
    typeLen = typeEnd-name;
    name = typeEnd+1;
} else if (!type) {
    return 0;
}

if (name >= nameEnd) {
    return 0;
}
nameLen = nameEnd-name;
\end{lstlisting}
ResTable::expandResourceRef开头的部分代码也是处理这个工作。
这里 包名跳过了'@',但是'+', '?', '*'未跳过,需要stringToValue
\cref{strtovalue}处理。

\subsubsection{ 资源项的id生成}
生成资源id有两种方法:Res_MAKEID宏、函数ResourceTable::makeResId 。
\begin{lstlisting}[caption={ResourceTypes.h Res_相关的宏定义}]
/**
 * Macros for building/splitting resource identifiers.
 */
#define Res_VALIDID(resid) (resid != 0)
#define Res_CHECKID(resid) ((resid&0xFFFF0000) != 0)

#define Res_MAKEID(package, type, entry) \
    (((package+1)<<24) | (((type+1)&0xFF)<<16) | (entry&0xFFFF))

#define Res_GETPACKAGE(id) ((id>>24)-1)
#define Res_GETTYPE(id) (((id>>16)&0xFF)-1)
#define Res_GETENTRY(id) (id&0xFFFF)

#define Res_INTERNALID(resid) ((resid&0xFFFF0000) != 0 && (resid&0xFF0000) == 0)
#define Res_MAKEINTERNAL(entry) (0x01000000 | (entry&0xFFFF))
#define Res_MAKEARRAY(entry) (0x02000000 | (entry&0xFFFF))

#define Res_MAXPACKAGE 255
\end{lstlisting}

\begin{lstlisting}[caption={ResourceTable.h的ResourceTable::makeResId}]
static inline uint32_t makeResId(uint32_t packageId,
                                 uint32_t typeId,
                                 uint32_t nameId)
{
    return nameId | (typeId<<16) | (packageId<<24);
}
\end{lstlisting}


ResTable::identifierForName  打印分析,只负责了   0x7f01xxxx(class attr),  0x7f08xxxx(class id), 0x7f0dxxxx (class  style)  以及 Android包中的内容。

\subsection{getBasePackageId}
\begin{lstlisting}
uint32_t ResTable::getBasePackageId(size_t idx) const
{
    if (mError != NO_ERROR) {
        return 0;
    }
    LOG_FATAL_IF(idx >= mPackageGroups.size(),
                 "Requested package index %d past package count %d",
                 (int)idx, (int)mPackageGroups.size());
    return mPackageGroups[idx]->id;
} 
\end{lstlisting}

\subsection{Accessor类}\label{accessorclass}
ResTable::Accessor (ResourceTypes.h) 声明了纯虚成员函数reportError,
ResourceTable (ResourceTable.h)继承 ResTable::Accessor。
Accessor 主要是为aapt 实现的,由于asset manger 库和aapt的交换。
Accessor 只传给 ResTable::stringToValue\cref{strtovalue}使用。

例如代码: 
\begin{lstlisting}[language=xml]
<TextView  android:id = "@idd/myTextView"     ... /> 
\end{lstlisting}

属性“android:id” 的值应该是引用 myTextView 的字符串,即"@id/myTextView",
这里多了一个"d",写成了"idd"。构建资源时就报
\\\indent“Error: No resource found that matches the given name (at 
"id" with value '@idd/myTextView')." 
\\\indent 此错误在函数 ResTable::StringToValue 报出,
它是调用了ResourceTable::reportError。 
类似的, getCustomResourceWithCreation 也是这个原理。


\subsection{getTableStringBlock}
android_content_AssetManager_loadResourceValue
\subsection{getResource / resolveReference}

lockBag

bag_entry 
    struct bag_entry {
        ssize_t stringBlock;
        ResTable_map map;
    };


\section{扩展标记语言文本}
ResXMLTree 是 ResXMLParser 的派生类,
ResXMLParser::next    -》 ResXMLParser::nextNode
ResXMLParser::indexOfAttribute , 比较简单,遍历元素的所有属性即可。
ResXMLParser::getAttributeStringValue

ResXMLTree 字面含义应该是个树形结构,其实是个流或者说数列,使用方法
\begin{lstlisting}
ResXMLTree block;
status_t err = parseXMLResource(in, &block, false, true);

if (err != NO_ERROR) {
        return err;
    }
block.next();

block.getElementName(&len);    //获取标记的名字
size_t len;
ssize_t itemIdentIdx = block.indexOfAttribute(NULL, "name"); //获取属性名name的属性索引
if (itemIdentIdx >= 0) {
    // 通过索引获取属性值
    itemIdent = String16(block.getAttributeStringValue(itemIdentIdx, &len));
} 
\end{lstlisting}

\subsection{声明文件的解析}
Android 的 AssetManager 对资源的解析中要用到资源 Resouces 类
PackageParser.java对 AndroidManifest.xml 的解析需要 AndroidManifestActivity
frameworks/base/core/res/res/values/attrs_manifest.xml

树形结构,树形解析
PackageParser.parseActivity  解析 activity 标记,并调用
PackageParser.parseIntent 解析 activity 标记下的 intent-filter

PreferenceInflater.onCreateCustomFromTag 调用 Intent.parseIntent 解析标记,
而 AliasActivity.parseAlias 也调用Intent.parseIntent 解析 intent 标记

PackageParser 里之所以不使用 Intent.parseIntent 解析 intent-filter
标签,是因为他希望解析结果以 ItentInfo 
对象保存,如ServiceIntentInfo、ActivityIntentInfo, 而 Intent.parseIntent
解析的结果是 Intent 对象。
但是有一点是相同的,即分层解析。
以 PackageParser 类为例:
\begin{enumerate*}
    \item 解析第一层标记
类的初始化函数 public Package parsePackage(File, String, DisplayMetrics,
int) ,主要作了两件事:
\begin{description*}
    \item[创建解析文件 AndroidManifest.xml 的解析器]
    parser = assmgr.openXmlResourceParser(cookie, "AndroidManifest.xml");
\item [调用解析函数解析]
    private Package parsePackage(Resources, XmlResourceParser, int, String[])
    这个函数解析的是 AndroidManifest.xml 中顶级标记 manifest
    下一层的所有标签,manifest 标记带有 package
    属性。比如,调用 parseApplication 对 application 标记解析; 调用
    parsePermission 解析标记 permission; 对标记 instrumentation
    使用方法 parseInstrumentation 分析。
    \end{description*}
\item 解析第二层标记
以 parseApplication 为例,对application 标记下的子标记 activity,service,provider等分别调用
parseActivity,parseService,parseProvider 解析。

\item 解析第三层
再以 parseActivity 为例,调用方法 parseIntent 和 parseMetaData 分别解析标记
intent-filter 与 meta-data。

\item 解析叶子标记
所谓叶子标记是其下没有了子标记,节点标记其下有子标记,也就是标记容器。
对于叶子标记的解析方法就是
Resources.obtainAttributes。对于节点标记,除了要获取属性 (
Resources.obtainAttributes ),还有一个重要的功能,就是拉出下一个标记
(XmlPullParser.next),如果拉出的标记深度 (XmlPullParser.getDepth)
大于容器标记深度,就解析,否则退出,这是一种递归算法,保证解析的标记都是同一个层次。
\end{enumerate*}

XmlPullParser
解析出来的是流,即按文件中出现的顺序解析节点,而不是节点的层次。采用层次处理解析的标记流,把不同层次的标记放在不同方法,同一个层次的都在一个方法,清晰而简洁。
\begin{lstlisting}[language=bash,]
platform$ find . -name "*.java" | xargs grep --color -n -C 0 "implements\s*XmlPullParser"
./libcore/xml/src/main/java/org/apache/harmony/xml/ExpatPullParser.java:34:public class ExpatPullParser implements XmlPullParser {
./libcore/xml/src/main/java/org/kxml2/io/KXmlParser.java:33:public class KXmlParser implements XmlPullParser {
\end{lstlisting}

\section{资源包装工具}
aapt

aapt p -M AndroidManifest.xml -S res -v

先目录,再文件。先文件,再xml。xml 先节点,再属性,及属性值。
函数调用顺序(main -> doPackage -> writeResourceSymbols -> writeSymbolClass ->
writeLayoutClssses -> getAttributeComment)

对AaptAssets.cpp 是处理目录、文件层次的工作,基本元素就是AaptFile,代表的是一
个文件,而AaptDir 是代表一个目录,可以包括子目录和文件,这是AaptFile在文件系统
的组织结构,而AaptGroup是Android的文件组织结构,他是把不同AaptDir(即目录)下
同名的文件组织在一起,这些文件代表了是不同配置下的同一个属性的不同取值。
而 AaptAssets是 AaptDir的派生类,这个结构是从apk的角度来看的,或者说是表示资源目录的根,
对于一个apk而言,其运行时不仅仅依赖于其中的 res目录和AndroidManifest.xml,也依赖于
frameworks-res.apk,甚至有覆盖(overlay)的apk包。
AaptGroupEntry 也是表示文件,不过附加了Android中的各种配置信息。

Command.c 的 doPackage 是创建apk的函数,其中的bundle(Bundle* )
主要是处理命令行参数的,是输入内容。
之后就是解析,主要是创建 AaptAssets 对象,然后调用 buildResources,
最后调用writeResourceSymbols、writeAPK输出文件。

doRemove Command.cpp
Bundle 是保存 aapt 命令行信息。
bundle->getFileSpecCount 命令行中包含的文件数,第一个文件名应该是压缩文件名
const char* fileName = bundle->getFileSpecEntry(i);
通过下标获取命令行中的文件名。


\begin{lstlisting}[caption=doPackage的核心代码,] 
// Load the assets.
assets = new AaptAssets();
err = assets->slurpFromArgs(bundle);
.. .. ..

// If they asked for any files that need to be compiled, do so.
if (bundle->getResourceSourceDirs().size() || 
    bundle->getAndroidManifestFile()) {
    err = buildResources(bundle, assets);
    .. .. ..
}
\end{lstlisting}

\tabcaption{aapt 主要源码一览}
\begin{tabular}{ll}
Main.cpp & 程序入口,解析命令行参数。\\
Commands.cpp & 执行aapt的各种命令,是处理框架。\\
Resource.cpp & apk 资源解析的处理模块,\\
    &全局函数及类PackageInfo, ResourceDirIterator实现\\
AaptAssets.cpp & AaptAssets,AaptDir,AaptGroup,\\
    &AaptFile,AaptGroupEntry类的实现\\
Package.cpp &  实现函数writeAPK。 \\
ResourceTable.cpp & ResourceTable, ResourceTable::Package,\\
    &ResourceTable::Type,ResourceTable::ConfigList,\\ 
    &ResourceTable::Entry, ResourceTable::Item,\\
    &ResourceFilter 等类的实现,\\
    &compileXmlFile, compileResourceFile 等函数的实现。\\
Images.cpp & preProcessImage, postProcessImage 实现,\\
&其他函数都是内部函数\\
XMLNode.cpp & XMLNode 类以及函数parseXMLResource实现\\
StringPool.cpp & StringPool 类的实现。\\
SourcePos.cpp & ErrorPos, SourcePos类的实现。\\
ZipEntry.cpp & ZipEntry, ZipEntry::LocalFileHeader, ZipEntry::CentralDirEntry
类实现\\
ZipFile.cpp & ZipFile, ZipFile::EndOfCentralDir类实现。\\
\end{tabular}

\subsection{主要数据结构}
\subsubsection{AaptDir}
AaptDir::slurpFullTree 先把当前目录下的文件和目录的路径全保存到fileNames,然后
遍历fileNames,若是目录,并已存入mDir,则从mDir取出
subdir(sp<AaptDir>),否则创建subdir,并对subdir调用slurpFullTree,最后把subdir存入mDir。
假如是文件,创建一个file(sp<AaptFile>),然后调用addLeafFile,addLeafFile 先以
文件名从mFiles 获取所对应的AaptGroup,如失败则创建AaptGroup,并和文件名一起加
入到mFiles,最后把file(sp<AaptFile>)加入到AaptGroup。这种情形应该是不同名录下
,同名的文件加入到一个AaptGroup,比如res目录各个 drawable*目录下同名的png文件
,又或者各个value下string.xml,等等。

\begin{lstlisting}
    String8 mLeaf;
    String8 mPath;

    DefaultKeyedVector<String8, sp<AaptGroup> > mFiles;
    DefaultKeyedVector<String8, sp<AaptDir> > mDirs;
\end{lstlisting}
AaptAssets.cpp 函数 isHidden 根据文件名把一些特定的目录和文件排除掉。
AaptAssets::slurpResourceTree 及 AaptDir::slurpFullTree 用到。
AaptAssets::slurpResourceTree 只被 AaptAssets::slurpFromArgs 调用到。

Resource.cpp 中定义的函数 collect_files

\subsubsection{AaptGroup}
AaptGroup : A group of related files (the same file, with different vendor/locale variations).
 他的一个主要数据结构是 :

\begin{lstlisting}
    String8 mLeaf;
    String8 mPath;

    DefaultKeyedVector<AaptGroupEntry, sp<AaptFile> > mFiles;
\end{lstlisting}
从这里可见, AaptGroup 是一系列 AaptFile 的有序集合,这些 AaptFile 以
AaptGroupEntry 的次序排列。为此 AaptGroupEntry 提供了 compare
来比较两个对象的大小,这个函数其实按一定顺序比较各个数据成员
(这个顺序其实就是数据成员的优先级)。

各个AaptGroup可以用字符串作为关键字,所以这些关键字也是这个AaptGroup的
一个属性或者说附加信息,而且在一个AaptDir内是唯一的:
\begin{lstlisting}[caption=AaptDir的属性]
    DefaultKeyedVector<String8, sp<AaptGroup> > mFiles;
\end{lstlisting}

\subsubsection{AaptAssets}
AaptAssets 是aapt中核心的数据结构,Images.cpp、AaptAssets.cpp、
Package.cpp、ResourceTable.cpp、Resource.cpp、XMLNode.cpp、Command.cpp 
等文件都见到她的身影。而相对来讲,AaptDir只是AaptAssets的基类,主要负责
处理文件收集、归类的一般性工作,AaptAssets.cpp、Package.cpp、Resource.cpp
、AaptAssets.h中有出现。

resDir 根据参数指定的目录名返回一个 AaptDir 智能指针,  resDirs()
返回的是一个列表,元素是 AaptDir 的智能指针

\subsubsection{AaptFile}
AaptFile 说是文件,其实更恰当的是一个buffer,他不关心 buffer
数据的格式、含义,只管申请空间、读、写数据。buffer相关的数据成员,如mData
即指向buffer, mDataSize 标示 mData 里使用的空间,mBufferSize 是 mData
总的空间 (从这里看出代码命名的混乱)。
虽然对buffer里面内容不关心,但是在buffer整体标示,AaptFile
还是有相关的数据成员,如mGroupEntry 是 AaptGroupEntry,mResourceType (String8),以及
原来的文件名、路径和数据压缩算法等相关信息。

\subsubsection{AaptGroupEntry}
AaptGroupEntry 作为 AaptFile 的关键字,就像String8 对于 AaptGroup,
AaptGroupEntry 也是每个 AaptFile的附加信息,在一个AaptGroup的各个
AaptFile可以以AaptGroupEntry区别,这就是说一个文件的不同配置下的具体内容,
如values-zh-rCN, values, values-zh-rTW目录中的strings.xml的标志:
\begin{lstlisting}[caption=AaptDir的属性]
DefaultKeyedVector<AaptGroupEntry, sp<AaptFile> > mFiles;
\end{lstlisting}


AaptGroupEntry 类似于 ResTable_config,  IMSI,
语言、屏幕大小、分辨率、尺寸规格、键盘等等,当然有版本号,
不然版本升级时由于数据不兼容而产生严重的错误。 为此
 AaptGroupEntry 提供了 compare 来比较两个对象的大小,
这个函数其实按一定顺序比较各个数据成员(这个顺序其实就
是数据成员的优先级)。而这些数据成员都是String8 类型的,
String8 类也提供了 compare 方法,他也就是调用了strcmp,
所以到底还是字符串比较。
\subsubsection{ResourceTypeSet}\label{restypeset}
资源类型
从 buildResources 中的代码看到:
\begin{lstlisting}
    sp<ResourceTypeSet> drawables;
    sp<ResourceTypeSet> layouts;
    sp<ResourceTypeSet> anims;
    sp<ResourceTypeSet> xmls;
    sp<ResourceTypeSet> raws;
    sp<ResourceTypeSet> colors;
    sp<ResourceTypeSet> menus;
\end{lstlisting}
 比如 
drawadble-hdpi/icon.png, drawable/icon.png, drawable-ldpi/icon.png
组成一个 AaptGroup, drawable-hdpi/background.png,
 drawable/background.png, drawable-ldpi/backgoround.png 组成有一个
AaptGroup, 这两个AaptGroup 均属于 drawable 资源类型。所有drawable
资源类型的 AaptGroup 就组成了一个 ResourceTypeSet。比如,xml 文件
也按这种方式组织。 
\subsubsection{ResourceDirIterator}\label{resdirit}
遍历一个ResourceTypeSet中包含的所有文件。
过程是遍历一个AaptGroup的AaptFile,再遍历下一个AaptGroup中的AaptFile,直至
把ResourceTypeSet中的所有AaptGroup遍历完毕。
核心是next(), 还提供了一系列获得当前文件的各种属性的函数。

\subsection{Resource.cpp 主要函数}
Resource.cpp\cite{1} 是解析、编译的主要源文件,包括入口,处理,输出,辅助等几类功能的
函数。
\subsubsection{buildResources}\label{buildres}
这是解析输入的核心函数。根据其代码里的注释,可以看到处理流程:
\begin{lstlisting}
 // First, look for a package file to parse.  This is required to
 // be able to generate the resource information.
 .. .. ..
 
// --------------------------------------------------------------
 // First, gather all resource information.
 // --------------------------------------------------------------
 .. .. ..  

 // --------------------------------------------------------------
 // Assignment of resource IDs and initial generation of resource table.
 // -------------------------------------------------------------


 // --------------------------------------------------------------
 // Finally, we can now we can compile XML files, which may 
 // reference resources.
 // -------------------------------------------------------------- 
 .. .. ..

// --------------------------------------------------------------
 // Generate the final resource table.
 // Re-flatten because we may have added new resource IDs
 // -------------------------------------------------------------- 
 .. .. ..
\end{lstlisting}
大概分五步:
\begin{itemize*}
    \item package file 就是指 AndroidManifest.xml,使用 parsePackage 解析
        获取包名(android:package 属性值),并以此初始化ResourceTable类的对象
        table\label{initresourcetbl},把asset(AaptAssets)加入到table中。
    \item 收集资源信息。
        \begin{itemize*}
            \item 集合文件。调用对asset调用 collect_files,结果放入 resources。
                asset 调用 setResources,保存resources。
            \item 遍历资源的覆盖,并收集文件。 AaptAssets 成员
                  mOverlay (sp<AaptAssets>), 假如他有赋值,则想上面一样处理,代码
                  稍有不同,结果是一样的。
                 因为都是 AaptAssets。而从 AaptAssets 角度,这个 mOverlay 也有
                 自己的 mOverlay,所以循环直至 mOverlay 为空。
\begin{lstlisting}
sp<AaptAssets> current = assets->getOverlay();
while(current.get()) { //current 是栈对象,不为空,get 说明有指向对象
    KeyedVector<String8, sp<ResourceTypeSet> > *resources 
       = new KeyedVector<String8, sp<ResourceTypeSet> >;
    current->setResources(resources);
    collect_files(current, resources);
    current = current->getOverlay();
}
\end{lstlisting}
            \item 对drawable,layout,anim,xml,raw,color,menu资源
                  类型调用applyFileOverlay,结构保存到drawables, 
                  layouts, anims, xmls, raws, colors, menus 变量。
            \item 对drawables 先调用 preProcessImages,再调用 
                  makeFileResources;对 layouts, anims, xmls, raws 
                  调用 makeFileResources\cref{makefileres}。
            \item 编译资源。 对 AaptAssets 及其 overlay 的values 类型AaptFile
                  调用compileResourceFile\cref{compileresfile}。
            \item 针对colors,menus 调用makeFileResources。

        \end{itemize*}

     \item  调用 table.assignResourceIds()\cref{assignresids}, 分配资源 IDs,生成初始资源表。
     \item  编译 XML 文件,因为这里涉及了交叉引用的问题。
        \begin{itemize*}
            \item 对 layout, anim, xml 资源类型的变量 layouts, anims, xmls
            使用compileXmlFile; 
            \item 对drawable的各个目录使用postProcessImages处理;
            \item 对 color,menu资源类型使用compileXmlFile;
            \item 编译 AndroidMainfest.xml 文件。
        \end{itemize*}
    \item 生成最终的资源表。把类R的符号加入table,并调用
        table.flatten(),因为可能要分配新的ID。 
\end{itemize*}
\subsubsection{makeFileResources}\label{makefileres}
makeFileResources 是以文件作为资源单位编译,使用 ResourceDirIterator
\cref{resdirit} 枚举ResourceTypeSet文件,对文件作三个步骤:
\begin{itemize*}
\item 对文件名作判断(文件名只能在[a-z0-9_.]中取值,所以大写字母是不行的),
\item 调用 table->addEntry\cref{tableaddentry}把文件加入到 ResourceTable 的一个Entry中,
\item assets->addResource\cref{aaptassetsaddres},把文件加入到 AaptAssets 
        中的相关数据,形成树形目录结构 。
\end{itemize*}

这个 Entry 的源位置是文件路径,行号为0, Package , type 就是
makeFileResources 最后参数,即目录名就是类型名, name 是文件名, value
是路径(根据平台,会把"$\backslash$" 转换为 "/")。               %\textbackslash
注意到 makeFileResources 的最后参数,她不仅仅作为目录名给
ResourceDirIterator 枚举文件, 而且在table->addEntry 作为 Type。

Type has an ConfigList , ConfigDescription
Entry has associated structure Item and SourcePos
SourcePos has only two member: filename, line number
For Id, which attribute name is “android:id”,  the finename is “<generated>”, and line number is 0.  
The following statement in function ResourceTable::Entry::generateAttributes tell us this information:
  status_t err = table->addEntry(SourcePos(String8("<generated>"), 0), package,
                                               id16, key, value);

 ResourceTable* table,   table->addEntry   .
                    Const sp<AaptAssets>\& assets  assets->addResource 
因为知道,估计 addEntry 是构建 内存 XML 模型树,Assets 是加入到。Apk包中的文件。  
从此入手 ,搜索"addEntry", compileResourceFile ()  , generateAttributes

\subsubsection{isValidResourceType}
\begin{lstlisting}
bool isValidResourceType(const String8& type)
{
    return type == "anim" || type == "drawable" || type == "layout"
        || type == "values" || type == "xml" || type == "raw"
        || type == "color" || type == "menu";
}
\end{lstlisting}

\subsubsection{parsePackage}
调用 parseXMLResource 处理 AndroidManifest.xml 文件,涉及节点"package"以及
"uses-sdk"。AaptAssets 根据 "package"的值调用 setPackage 。

buildResources 是其唯一被调用的地方。
从其中代码可以,AaptGroup 的是同名文件的集合,对于 "AndroidManifest.xml"只需取其第一个即可。
\begin{lstlisting}
status_t buildResources(Bundle* bundle, const sp<AaptAssets>& assets)
{
    // First, look for a package file to parse.  This is required to
    // be able to generate the resource information.
    sp<AaptGroup> androidManifestFile =
            assets->getFiles().valueFor(String8("AndroidManifest.xml"));
    if (androidManifestFile == NULL) {
        fprintf(stderr, "ERROR: No AndroidManifest.xml file found.\n");
        return UNKNOWN_ERROR;
    }

    status_t err = parsePackage(bundle, assets, androidManifestFile);
    .. .. ..
}

static status_t parsePackage(Bundle* bundle, const sp<AaptAssets>& assets, const sp<AaptGroup>& grp)
{
    .. .. ..
    sp<AaptFile> file = grp->getFiles().valueAt(0);

    ResXMLTree block;
    status_t err = parseXMLResource(file, &block);
    .. .. ..
}
\end{lstlisting}
\subsubsection{collect_files}
这个函数就是把AaptDir 中的 AaptGroup 中按资源类型
(ResourceType)组织为ResourceTypeSet\cref{restypeset}。
不同的ResourceTypeSet 以链表resources
(KeyedVector<String8, sp<ResourceTypeSet> >*)
保存在一起。

他遍历 AaptDir 的各个 AaptGroup,然后取出AaptFile的链表
files,由于 AaptGroup 中保存的 AaptFile 的mResourceType 是一样的,
所以就对第一个AaptFile 调用 getResourceType()获取资源类型。
获取资源类型资源,查看 resources 中有无此资源类型,无则创建资源
类型set, 把AaptGroup 加入到set,再把set挂到列表resources。若已有,
则查看资源类型有无相同的AaptGroup,这是根据 AaptGroup 的 mLeaf (即文件名)
,没有AaptGroup 就简单,把他加入到 set 中即可;如果有的话,把新的 AaptGroup
的AaptFile 加入到已存在的 AaptGroup 中。

DefaultKeyedVector<AaptGroupEntry, sp<AaptFile>

\subsubsection{getResourceFile}
getResourceFile 就是为了获取 resources.arsc。
(assets->getFiles().valueFor(String8("resources.arsc"));)因为 AaptAssets 继承
 AaptDir, getFiles 是后者实现的,也就是说在AaptAssets的根目录下找resources.arsc 
这个文件,没有的话就加入 resources.arsc 这个节点
(assets->addFile(String8("resources.arsc"), AaptGroupEntry(), String8(), NULL, String8());)
当然也是在根目录下, addFile 也是类 AaptDir的接口

\subsubsection{postProcessImages}
postProcessImage 处理 drawable 目录。
\subsubsection{writeResourceSymbols}
	writeResourceSymbols  生成文件 R.java 及 Manifest.java,
这两个文件中的头部声明"/* AUTO-GENERATED FILE.  DO NOT MODIFY. ..."
就是由这个函数打印的,
但是她仅仅是合成文件名和路径,打印头部声明,真正干活还要靠
writeSymbolClass ,即 writeResourceSymbols 调用了 writeSymbolClass,
这两个函数均在 Resource.cpp 实现。

因为可以看到	AUTO-GENERATED FILE.  DO NOT MODIFY.  头部内容
writeSymbolClass 是干活。
\subsubsection{writeSymbolClass}

\subsection{ResourceTable.cpp 的主要函数}
强调一下,ResourceTable 的对象是在 buildResources 函数开始创建的,别的地方
没有创建他的。
\subsubsection{compileXmlFile}
compileXmlFile 是以文件为单位编译的,编译后还是一个文件,
但不再是文本文件, 而是二进制文件, APK 资源的目录结构和
源代码中的一样,即编译前后文件的相对路径(从 res 开始)没
改变。 R.java 相对应的资源是 resources.arsc,这里当然有以
文件为单位的资源索引,也有像 resources 标记下的各种资源。

\begin{lstlisting}
  status_t err = root->parseValues(assets, table);
  err = root->flatten(target,
            (options&XML_COMPILE_STRIP_COMMENTS) != 0,
            (options&XML_COMPILE_STRIP_RAW_VALUES) != 0);
            
  //设置文件的压缩算法
  target->setCompressionMethod(ZipEntry::kCompressDeflated);
        
  assignResourceIds    
   
  generateAttributes
  setIndex    setEntryIndex
  sp<Entry>::assignResourceIds   
\end{lstlisting}

\subsubsection{compileResourceFile}\label{compileresfile}
compileResourceFile  从函数体开头部分字符串的定义可知,其解析 <resources/> 标签下的元素,
\begin{itemize*}
    \item Top-level tag :                     resources 
    \item Identifier declaration tags :       declare-styleable, attr
    \item Data creation organizational tags : string, drawable, color, bool,
        integer, dimen, fraction, style, plurals, array, string-array,
        integer-array, public-padding, private-symbols, add-resource, skip,
        eat-comment
    \item Data creation tags :       bag, item
    \item Attribute type constants : enum 
    \item plural values :            other, zero, one, two, few, many, 
    \item useful attribute names and special values:  name, translatable, false
\end{itemize*}
最后两个不是tag,而是属性名和值

\begin{lstlisting}[language=xml,label=xmlexam2,caption={Android XML 举例(续)}]
<?xml version="1.0" encoding="utf-8"?>
<resources>
  <dimen name="title_texture_width">120px</dimen>
  <color name="delete_color_filter">#A5FF0000</color>  
  <integer name="config_allAppsBatchSize">0</integer>
  <bool name="lockscreen_isPortrait">true</bool>

  <style name="AnimationPreview">
      <item name="android:windowEnterAnimation">@anim/fade_in_fast</item>
      <item name="android:windowExitAnimation">@anim/fade_out_fast</item>
  </style>

  <string-array name="menu_task_operation">
  	<item>@string/switch</item>
  	<item>@string/uninstall</item>
  </string-array>

  <attr name="direction">
      <enum name="vertical" value="0" />
      <enum name="horizontal" value="1" />
  </attr>

  <skip />
  <style name="Widget.KeyboardView" parent="android:Widget">
      <item name="android:background">@android:drawable/keyboard_background</item>
      <item name="android:keyBackground">@android:drawable/btn_keyboard_key</item>
      <item name="android:verticalCorrection">-10dip</item>
      <item name="android:shadowColor">#BB000000</item>
  </style>

  <declare-styleable name="Favorite">
      <attr name="className" format="string" />
      <attr name="title" format="reference" />
  </declare-styleable>
  <declare-styleable name="AndroidManifestApplication" parent="AndroidManifest">
      <attr name="name" />
      <attr name="theme" />
      <attr name="label" />
      <attr name="icon" />
      .. .. ..
  </declare-styleable>
  <private-symbols package="com.android.internal" />
  <public type="attr" name="screenDensity" id="0x010102cb" />
  <public type="drawable" name="presence_video_away" id="0x010800ac" />
  <public type="style" name="Theme.Wallpaper.NoTitleBar" id="0x0103005f" />

\end{lstlisting}
compileResourceFile 函数比较庞大,有 800 多行, 但是结构还是简单的。
原理跟 PackageParser.java 解析 AndroidManifest.xml
一样,即把标记流按层处理,这也是流-拉模型的典型处理方法了。下面简单分析一下:
\begin{itemize*}
    \item 定义标签变量\\基本上变量名就是标签名加上16,标签名有'-',变量名换为'
        _',比如 string16 表示标签"string"。
\begin{lstlisting}
const String16 skip16("skip");
const String16 eat_comment16("eat-comment");

// Data creation tags.
const String16 bag16("bag");
const String16 item16("item");
\end{lstlisting}
    \item parseXMLResource\cref{parsexmlres}\\
这个函数比较简单,现在不清楚具体做什么,但可以肯定的是她调用 XML
解析器引擎初始化 ResXMLTree 结构。
    \item 拉出 resources 标记\\
跳过所有的(命名空间)声明后,判断是否为标记,然后检查标记名
\begin{lstlisting}
if (strcmp16(block.getElementName(&len), resources16.string()) != 0) {
    ... ...
    return UNKNOWN_ERROR;
}
\end{lstlisting}
resources16 就是值为 "resources" 的常量对象。
\item 处理 resources 标记\\
    这个语句块的结构如下:
\begin{lstlisting}
while ((code=block.next()) != ResXMLTree::END_DOCUMENT && code != ResXMLTree::BAD_DOCUMENT) {
    if (code == ResXMLTree::START_TAG) {
        ... ...    // 处理tag语句块
    }
    else if (code == ResXMLTree::START_NAMESPACE || code == ResXMLTree::END_NAMESPACE) {
    }
    else if (code == ResXMLTree::TEXT) {
        if (isWhitespace(block.getText(&len))) {
            continue;
        }
        ... ...
        return UNKNOWN_ERROR;
    }
}
\end{lstlisting}
\end{itemize*}

处理 tag 语句块基本结构就是 if-else if-else 的控制语句,判断条件就是 tag
名,tag处理一般形式如下:
\begin{lstlisting}
if (strcmp16(block.getElementName(&len), skip16.string()) == 0) {
       ... ... // 子标记解析语句块  
       while ((code=block.next()) != ResXMLTree::END_DOCUMENT
               && code != ResXMLTree::BAD_DOCUMENT) {
           if (code == ResXMLTree::END_TAG) {
               if (strcmp16(block.getElementName(&len), skip16.string()) == 0) {
                   break;
               }
           }
           else if (code == ResXMLTree::START_TAG) {
               ... ... //子标记中的子标记处理语句
           }
       }
       continue;

   } else if () {
       ......
   }
\end{lstlisting}
针对每个标签要给下列的值赋值:
curTag ,curType ,curFormat ,curIsBag ,curIsBagReplaceOnOverwrite ,curIsStyled 
,curIsPseudolocalizable ,curIsFormatted localHasErrors 

skip16、eat_comment16 最简单的形式,继续拉,直至标记结束,没有解析语句,因为这
两标签是注释的。同样是xml的注释,放在这两个标签里的注释是不会出现在R.java 或 Manifest.java
\par
public16, public_padding16, private_symbols16,add_resource16,
有子标记解析语句块;
\par attr16 调用了 compileAttribute 处理。
\par declare_styleable16 标记处理比较麻烦,因为底下又子标记,
所以有子标记中的子标记处理语句,这些标记有
skip16, eat_comment16, attr16 。(addSymbol, appendComment) 
以上这些标签处理完了直接continue, 执行下一次大循环。
\par
item16,string16,drawable16,color16, bool16, integer16, dimen16,
fraction16 以及 bag16, style16, plurals16, array16, string_array16,
integer_array16
根据标记名仅仅作了少量处理,比如设置了一些变量名,然后放在后面统一处理。他们都有一个共通的特点,就是有"name"属性。
但自bag16后的标记,需要是个容器或者说袋子,里面还有 item16
标记要处理,与 declare_styleable16 ,处理大概流程就是上面的代码片断,完了调用 parseAndAddBag
。前面的其他标记处理就比较简单,调用 parseAndAddEntry 即可。

因为对这些标签处理都是比较类似的,在他们的if/else if 里面只是
修改了 curTag ,curType ,curFormat ,curIsBag 值,然后统一处理,
不想上面的,直接continue,跳到大的while循环。
对数组(array, string16-array)、复数、style 都是bag,是个容器,要
确定其父节点:
\begin{lstlisting}
String16 ident;
ssize_t identIdx = block.indexOfAttribute(NULL, "name");
if (identIdx >= 0) {
    ident = String16(block.getAttributeStringValue(identIdx, &len));
} else {
.. .. ..
// Figure out the parent of this bag...
String16 parentIdent;
ssize_t parentIdentIdx = block.indexOfAttribute(NULL, "parent");
if (parentIdentIdx >= 0) {
    parentIdent = String16(block.getAttributeStringValue(parentIdentIdx, &len));
} else {
    ssize_t sep = ident.findLast('.');
    if (sep >= 0) {
        parentIdent.setTo(ident, sep);
    }
}
\end{lstlisting}
bag 的父节点先是查找属性parent的值,若无则分析属性name的值,其值"."之前的部分
就是其父节点。
\begin{lstlisting}
<style name="Theme.Dialog.AppError">
<style name="Theme.SearchBar" parent="Theme.Panel">
\end{lstlisting}
第一种,父节点就是Theme.Dialog的style。第二种,父节点是Theme.Panel的style。

\subsubsection{compileAttribute}
compileAttribute(R.java declare-style的属性的产生)

\subsubsection{ResourceTable:addIncludedResources}
回过头看,buildResources 开头的一条语句 
\begin{lstlisting}
err = table.addIncludedResources(bundle, assets);
\end{lstlisting}
 ResourceTable::addIncludedResources 函数的最后面调用 ResourceTable::getType
 建立了 attr 类型,即 attr 类,这就是 R.java 的开头总是能看到内部类attr,即使是空的。

\subsubsection{Package类和ResourceTable::getPackage}\label{getpackage}
\begin{lstlisting}
            mHaveAppPackage = true;
            p = new Package(package, 127);
        } else {
            p = new Package(package, mNextPackageId);
        }
\end{lstlisting}
127 即0x7F, 这是R.java常见的0x7fxxxxxx的来源吧? Entry.
Package 表示一个apk包,或者说是R类。


ResourceTable::getPackage 跟 mNextPackageId 关系不大, mNextPackageId 为2
“*** NEW PACKAGE: "com.android.speechrecorder" id=127”
ResourceTable::mNextPackageId 与分配的Package ID 关系密切,
ResourceTable::ResourceTable 构造函数把 此值初始化为1。  
Package ID 的值一般只有两个 \\
\begin{tabular}{ll}
1  & →  android \\
127 & → com.android.launcher\\
\end{tabular}\\
android、com.android.launcher 是 AndroidManifest.xml 的 "manifest"标记的
package属性值,只不过一个是 frameworks/base/core/res下的,表示Android系统的,
一个是普通应用程序的。

仅在ResourceTable.cpp模块内部使用, 由getPackage 创建Package 对象。
搜索\verb|"new\s*Package"|即可定位创建对象的地方。

\subsubsection{Type类和ResourceTable::getType}\label{restblgtype}
ResourceTable::Package::getType 包装, 先在mType查找,找不到创建一个Type对象。
获取 Type,getType 查询 Package 的 mTypes 里有没有这个Type, 查询没有就建立一个。

Type 表示一个R的内部类。
Type 不直接包含 Entry, 而是包含了 ConfigList
\begin{lstlisting}
SortedVector<ConfigDescription> mUniqueConfigs;
DefaultKeyedVector<String16, sp<ConfigList> > mConfigs;
Vector<sp<ConfigList> > mOrderedConfigs;
\end{lstlisting}
ConfigList 才拥有 Entry:
\begin{lstlisting}
DefaultKeyedVector<ConfigDescription, sp<Entry> > mEntries;
\end{lstlisting}
也就是说同一个Type的Entry按配置归类,即同一个文件名在有不同配置,
他们都放在同一个 ConfigList,这就是相当于 AaptGroup。
ConfigDescription 相当于 AaptGroupEntry, Entry 相当于 AaptFile。
ConfigDescription 是 ResTable_config 的派生类,但没有增加数据成员,
ResTable_config 对配置的描述的用union,每个union的域是基本数据类型,
如uint8_t, uint16_t, uint32_t,面向机器;而AaptGroupEntry 中对配置的描述都是
String8,即字符串,面向程序员。

\subsubsection{Entry类和ResourceTable::addEntry}\label{tableaddentry}
他实际上就是ResourceTable::getEntry 包装,后者调用getType
\cref{restblgtype}获得对应的Type,再调用 Type::getEntry
获取 Entry。 先查询,若无就new 一个 Entry。
Entry 表示R的内部类的一个属性。

\subsubsection{ResourceTable::getResId}\label{getresid}
 这个函数名对应了三个函数,其获取资源id的过程类似于stringToValue 对引用的解析
\cref{derefer} 的三部曲: expandResourceRef, identifierForName, 获取资源id。 
最原始的
\begin{lstlisting}
uint32_t ResourceTable::getResId(const String16& ref,
                                 const String16* defType,
                                 const String16* defPackage,
                                 const char** outErrorMsg,
                                 bool onlyPublic) const
\end{lstlisting}
只知道资源项的引用(即完整的资源名),不知到具体类型和包,
需要调用ResTable::expandResourceRef解析出来,这里注意defPackage在调用时赋为NULL,
那么,传给expandResourceRef 就是 mAssetsPackage,后者是ResourceTable初始化时赋
值的\cref{initresourcetbl}。然后他调用上一级的:
\begin{lstlisting}
uint32_t ResourceTable::getResId(const String16& package,
                                 const String16& type,
                                 const String16& name,
                                 bool onlyPublic) const
\end{lstlisting}
这里知道了包名(package)、类型(type)、资源名(name),
他先资源名去查所包含的资源池里有无此资源项
\begin{lstlisting}
    uint32_t rid = mAssets->getIncludedResources()
        .identifierForName(name.string(), name.size(),
                           type.string(), type.size(),
                           package.string(), package.size(),
                           &specFlags);
\end{lstlisting}
此后的处理流程也是和 stringToValue 对引用的解析 \cref{derefer}类似的。

如果这个步骤获取不到, 即所包含的资源池没有此资源,在当前解析的资源包里找,
其原理同 ResourceTable::getCustomResource,虽然getResId 自己写了代码,但是
一致的。

\begin{lstlisting}
inline uint32_t ResourceTable::getResId(const sp<Package>& p,
                                        const sp<Type>& t,
                                        uint32_t nameId)
{
    return makeResId(p->getAssignedId(), t->getIndex(), nameId);
}
\end{lstlisting}
makeResId 就是把三个数值移位,位或运算。


\subsubsection{ResourceTable::getCustomResourceWithCreation}\label{gcrwc}
这个函数会创建新的资源项(Entry),并返回id
\begin{itemize*}
\item getCustomResource
\item addEntry
\item getResId
\end{itemize*}
Custom 就是自定义的意思,针对 “@+”这种。
创建资源主要是 addEntry。
\subsubsection{ResourceTable::assignResourceIds}\label{assignresids}
此函数只被buildResources\cref{buildres} 调用。
ResourceTable::Entry::assignResourceIds
ResourceTable::Package::getAssignedId
ResourceTable::Package:: mIncludedId 只在对象初始化时涉及到

\subsubsection{ResourceTable::flatten}\label{restblflatten}

\subsection{XMLNode.cpp的主要函数}
\subsubsection{parseXMLResource}\label{parsexmlres}
parseXMLResource 初始化 XML 解析。 因为 parseXMLResource 调用方法
XMLNode::parse, 后者正是调用 XML_Parser 解析器引擎。
XMLNode::parse 通过读取 XML
文件把标记保存到内存中,解析器的用户数据是一个 ParseState
类型结构体 (她是 XMLNode 类的私有结构体)变量,解析出来的数据保存在成员stack
(这是一个Vector对象,元素是sp<XMLNode>),结构体成员 root (sp<XMLNode>
类型)是指向 stack 中的第一个节点。 TODO::
parseXMLResource   转化到 XMLNode 树形结构, 再到 AaptFile,最后到 ResXMLTree
结构。从XML 的树形-》 Expat 的流 -》 XMLNode 的树 -》 ResXMLTree

\subsection{主要输出函数和数据结构}

\subsubsection{writePublicDefinitions}
-P  选项(不是-p[ackage])   指定的文件,  
<public type="string" name="group_applications" id="0x7f0c000f" />

-G out/target/common/obj/APPS/Launcher2_intermediates/proguard_options

\subsection{Package.cpp主要函数}
\subsubsection{writeAPK}
Package.cpp定义的,生成.apk, APK的文件名由getOutputAPKFile获取
TODO::可以比较 writeAPK 与 writeSymbolClass 的异同啊!!!

创建一个 ZipFile  文件zip,先由processAssets 对AaptAssets处理,把结果放入
zip中,然后,processJarFiles 再对zip进行处理,最后,删除不必要的文件。

\subsection{AaptAssets.cpp}
\subsubsection{AaptDir::makeDir}
\begin{lstlisting}
sp<AaptDir> AaptDir::makeDir(const String8& path)
{
    String8 name;
    String8 remain = path;

    sp<AaptDir> subdir = this;
    while (name = remain.walkPath(&remain), remain != "") {
        subdir = subdir->makeDir(name);
    }

    ssize_t i = subdir->mDirs.indexOfKey(name);
    if (i >= 0) {
        return subdir->mDirs.valueAt(i);
    }
    sp<AaptDir> dir = new AaptDir(name, subdir->mPath.appendPathCopy(name));
    subdir->mDirs.add(name, dir);
    return dir;
}
\end{lstlisting}
subdir 初始值是 this。
对于path为"res",如果未对其makeDir,while、if两个判断都不成立,创建一个AaptDir,
并加入到subdir; 以后对其直接makeDir, 在if中返回。 对于 path为 "res/layout/mypng",初次对其直接makeDir,在while处,第一次name为"res",
递归调用返回"res"所在的AaptDir,这时subdir换成了"res"所在的;第二次,name为"layout",
while,if为假,新建一个AaptDir,并返回,subdir为"layout"所在的目录,while,if不成立,
新建AaptDir;两次循环后,name为"mypng",while 条件不成立,
循环结束;最先的makeDir才走到if判断,这时 name 为 mypng, if不成立成立。 
第二次及其后面,对"res/layout"直接makeDir,while循环的makeDir调用中 while不成立,if条件
成立。

对于 while 要说明一下,发起调用的makeDir的执行中,他要等待while循环执行完毕,才能往下执行。
而由 while 里发起的makeDir执行中,不管是发起的、还是发起的while中的subdir等,其while循环
都不会被执行,因为其path参数不含有"/"。 发起的调用在while循环执行结束后(只要其进入了
while ),subdir 是倒数次级,或是主目录,如 path 为 "res"或"res/drawable",subdir就指向"res", 
paht为"res/drawable/mypng", subdir 就是 "drawable"。 这个函数执行的结果
就是形成跟目录一样的树形结构。 并且返回最末端(最深)的目录所在的目录,如果没有就是他自己。

对于不存在的目录foo/bar/a,通过while循环中的递归,迭代最后,while及if判断都不成立,执行最后的新建AaptDir 对象,最后
效果同mkdir -p foo/bar/a。 如果 foo/bar/a已存在,那么最后递归到if返回,前面的递归也到if止。所以,while循环可以保证目录创建。
其最大递归深度与目录的路径深度关联。

while 与 if 的先后顺序可否对换?
\subsubsection{AaptAssets::addResource}\label{aaptassetsaddres}
\begin{lstlisting}
void AaptAssets::addResource(const String8& leafName, const String8& path,
                const sp<AaptFile>& file, const String8& resType)
{
    sp<AaptDir> res = AaptDir::makeDir(kResString);
    String8 dirname = file->getGroupEntry().toDirName(resType);
    sp<AaptDir> subdir = res->makeDir(dirname);
    sp<AaptGroup> grr = new AaptGroup(leafName, path);
    grr->addFile(file);

    subdir->addFile(leafName, grr); 
}
\end{lstlisting}
 第一行,由于 AaptAssets 是AaptDir 的派生类,所以可以调用父类的 makeDir,
kResString 是全局的,指向 "res" 目录。
所以,res 就是代表目录"res"。 

两个疑问: dirname ,包含“res”?
每个AaptGroup只有一个文件? 从addResource 看是这样子的。后续有无合并的动作?

\subsection{注释}
R.java 里的注释
函数 compileAttribute、compileResourceFile, mayOrMust, 以及数组 gFormatFlags
均会给节点增加一点注释。 而 declare-styleable 声明属性时,生成的 attr、
styleable 内部类及相关的域,会有大量的注释,他们是有 writeLayoutClasses 搞的。

appendComment

\subsection{属性}
只有在编译时检测到的属性才能在XML layout文件中设置。标签的值列表参见:
frameworks/base/core/res/res/values/public.xml
frameworks/base/api/*.xml 

XMLNode.cpp 定义了几个常见的命名空间(NameSpace) 
XLIFF_XMLNS      RESOURCE_ROOT_NAMESPACE  
上面的标记会有很多的属性,这些属性怎么来的呢? 这些属性都是由
declare-styleable 声明的,Android 系统级的属性,
参见 SDK 文档
docs/reference/android/package-summary.html
docs/reference/android/R.id.html
或者编译后的文档
out/target/common/docs/offline-sdk/reference/android/package-summary.html
out/target/common/docs/offline-sdk/reference/android/R.id.html
应用中,我们一般把声明属性的 declare-styleable 标记放在文件
res/values/attrs.xml,而使用这些属性的 XML 文件不能与 attrs.xml
同目录,即不能放在 res/values,可以放在 res/xml, res/color, res/anim 等等,或者
res/drawable 也行?

parseStyledString 
递归函数

 后者调用调用 compileAttribute (R.java declare-style的属性的产生)
解析文件中的各个标签。

\subsection{多语言}
string16 中可能与多语言相关的函数 translatable16, getAttributeStringValue,
getResourceSourceDirs。

\subsection{覆盖}
\begin{lstlisting}
static bool applyFileOverlay(Bundle *bundle,
                             const sp<AaptAssets>& assets,
                             sp<ResourceTypeSet> *baseSet,
                             const char *resType)

    // look for same flavor.  For a given file (strings.xml, for example)
    // there may be a locale specific or other flavors - we want to match
    // the same flavor.
\end{lstlisting}

\subsection{文件压缩}
ZipFile* zip = openReadWrite(zipFileName, false);
这个是打开 apk 压缩文件,apk 采用的压缩格式其实就是Zip算法。
ZipEntry 就是压缩文件的单位文件的数据结构。

\begin{lstlisting}
ZipEntry* entry = zip->getEntryByName(fileName);

zip->remove(entry);                                        //删除 
zip->add(fileName, bundle->getCompressionMethod(), NULL);   //增加 
zip->addGzip     //增加 .gz 文件 
zip->flush(); 
delete zip;
\end{lstlisting}

\begin{comment}


frameworks/base/libs/utils/Asset.cpp      _CompressedAsset::getBuffer   
frameworks/base/include/utils/Asset.h      UNCOMPRESS_DATA_MAX 


ResourceTable.cpp  函数 compileAttribute、compileResourceFile、compileXmlFile 
compileResourceFile、parsePackage 等解析 XML 文件的函数都要调用

platform$ find sdk | xargs grep --color -n -e "android.libraries.res"
sdk/anttasks/src/com/android/ant/SetupTask.java:551:            antProject.addReference("android.libraries.res", resPath);
sdk/anttasks/src/com/android/ant/AaptExecLoopTask.java:368:        Object libSrc = taskProject.getReference("android.libraries.res");
sdk/anttasks/src/com/android/ant/AaptExecLoopTask.java:399:        Object libPath = taskProject.getReference("android.libraries.res");

sdk/anttasks/src/com/android/ant/AaptExecLoopTask.java:301:     private void callAapt(String customPackage) {
  \end{comment}

\appendix
%\appendixpage
\addcontentsline{toc}{section}{附录}\markboth{APPENDICES}{}
\begin{appendices} 
\end{appendices}

%加一道横线
\noindent\rule[-1ex]{\textwidth}{3pt}\\[2.0ex] 
\addcontentsline{toc}{section}{参考文件}
\begin{thebibliography}{主要代码}
    %\bibitem {android的binder驱动} \href{http://blog.sina.com.cn/s/blog_69f669470100s40o.html}{android的binder驱动 进程,线程,线程池},
    \bibitem{1}frameworks/base/libs/utils/AssetManager.cpp
    \bibitem{2}frameworks/base/core/jni/android_util_AssetManager.cpp,
    \bibitem{3}frameworks/base/core/jni/android_util_XmlBlock.cpp,
    \bibitem{4}frameworks/base/core/jni/android_util_StringBlock.cpp
    \bibitem{5}frameworks/base/core/java/android/content/res/AssetManager.java
    \bibitem{6}frameworks/base/core/java/android/content/pm/PackageParser.java
    \bibitem{7}frameworks/base/core/java/android/content/pm/PackageManager.java
    \bibitem{8}frameworks/base/native/android/asset_manager.cpp
    \bibitem{9}
\end{thebibliography}
\printindex
\end{document}
